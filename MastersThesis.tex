\documentclass[11pt,DIV12,BCOR8,openany,a4paper]{book}

% use this declaration to set specific page margins
% \usepackage[a4paper , twoside, lmargin = {2.7cm} , rmargin = {2.9cm} , tmargin = {2.7cm} , bmargin = {4.6cm} ]{geometry}
\usepackage[a4paper, twoside, bindingoffset=1cm, head=17pt]{geometry}
%\usepackage[a4paper]{geometry}
\usepackage[numbers]{natbib} %numbers for numerical citations https://www.overleaf.com/learn/latex/Bibliography_management_with_natbib
% \usepackage[title]{appendix}
\usepackage[nottoc]{tocbibind} % include bibliography in table of contents. options: [nottoc, notlof, notlot]
\usepackage[english]{babel}
\usepackage{bibgerm}       		% german references
\usepackage[utf8x]{inputenc} % german characters
\usepackage{graphicx} 				% it's recommended to use PDF images but you can use JPG or PNG as well
\usepackage{url}           		% format URLs
\usepackage{hyperref} 				% create hyperlinks
\usepackage{epigraph}       % inspirational quotes
\usepackage{dirtytalk} % inline quotes
\usepackage{listings}	% for source code
\lstset{frameround=fttt,language=Python,numbers=left,breaklines=true}
\usepackage{subfig}						% two figures next to each other (example: figure 3a), figure 3b)
% \usepackage{scrpage2}					% header and footer line
% \setlength{\headheight}{1.1\baselineskip} 

% % header and footer line - no header & footer line on pages where a new chapter starts
% \pagestyle{scrheadings}
% %\ohead{Automatic Service Generation}
% \ohead{\headmark}
% \automark{chapter}
% \ihead{}
% % \ihead{\headmark}
% % \automark{section}
% % \cfoot[]{\thepage}
% \ifoot[]{\thepage}

\usepackage[
% Breite anpassen.
headwidth=(\the\textwidth+14mm):-7mm:7mm,
footwidth=(\the\textwidth+14mm):-7mm:7mm,
% Dicke der Linien festlegen.
headsepline=0.5pt,
footsepline=0.5pt,
% Linien auch im Seitenstil plain anzeigen.
plainheadsepline=true,
plainfootsepline=true
]{scrlayer-scrpage}
% Alle Inhalte löschen.
\clearpairofpagestyles
% Schriftformatierung zurücksetzen.
\setkomafont{pageheadfoot}{}
% Linien einfärben.
\addtokomafont{headsepline}{\color{gray}}
\addtokomafont{footsepline}{\color{gray}}
% Statische Inhalte.
\ohead{\headmark}
\ofoot*{\pagemark}
% Unterschiedliche Inhalte für gerade/ungerade.
% \refoot*{Thorsten Frommen}
% \lofoot*{Mein Superduperbuch}
    % Inhalte festlegen.
\automark[chapter]{chapter}
\automark*[section]{}
% Eigener Seitenstil (keine Inhalte, aber Linien).
\newpairofpagestyles{preface}{
\clearpairofpagestyles
}

%\ifoot{Future Smart Cam, TU Berlin, Fachgebiet AV, 2009}

% set path where images are stored
\graphicspath{{./img/}}

% \input{./misc/hyphenation} 					% use this file to set explicit hyphenations (doesn't seem to work correctly)
\hyphenation{RecipeIdType}
%% Useful packages
% \usepackage{showframe}
\usepackage{geometry}
\usepackage{diagbox}
\usepackage{amsmath}
\usepackage{graphicx}
\usepackage[colorinlistoftodos]{todonotes}
%\usepackage[colorlinks=true, allcolors=blue]{hyperref}
\usepackage{lscape}
\usepackage{url}
\usepackage{minted}
\usepackage{tabu}
\usepackage{enumitem}
\usepackage{hyperref}
\usepackage{color, colortbl}
\usepackage{multirow}
\usepackage{longtable,array,ragged2e}
\newcolumntype{P}[1]{>{\RaggedRight\arraybackslash\hspace{0pt}}p{#1}}
\usepackage{eurosym}
\usepackage[]{acronym}
\usepackage{setspace}
\definecolor{mypink1}{rgb}{0.858, 0.188, 0.478}
\definecolor{Gray}{gray}{0.9}

%morphological box
\usepackage{booktabs}
\usepackage{tikz}
\usetikzlibrary{matrix}

\newcommand{\zeilenabstand}{\normalbaselineskip}

\newcommand\grafik[2]{%
  \begin{minipage}{0.1cm}
    \centering\smash{\raisebox{\tabcolsep}{#1}}%
    \includegraphics[width=\linewidth]{#2}%
  \end{minipage}%
}

\tikzset{vp/.style={circle,fill,inner sep=3pt}}
\newcommand\verbindungslinie[3]{
  \foreach [remember=\p as \lastp (initially #2)] \p in {#3}
    \draw[#1](\lastp.center)node[vp]{}--(\p.center)node[vp]{};
}
%end morphological box

% proto files. See https://github.com/aytchell/latex-listings-protobuf
\newcommand{\SetProtoColorsBlueish}{
  % Colors inspired by the NASM style of Robin Eklind
  % https://github.com/mewspring/latex
  \definecolor{proto_basic}{RGB}{0,0,0}             % black
  \definecolor{proto_keyword}{RGB}{0,0,255}         % blue
  \definecolor{proto_type}{RGB}{128,0,0}            % dark red
  \definecolor{proto_options}{RGB}{128,0,128}       % purple
  \definecolor{proto_comment}{RGB}{0,128,0}         % dark green
  \definecolor{proto_string}{RGB}{255,0,0}          % red
  \definecolor{proto_number}{RGB}{108,113,196}      % violet
  \definecolor{proto_ident}{RGB}{0,0,0}             % black
  \definecolor{proto_digits}{RGB}{0,0,128}          % dark blue
  \definecolor{proto_background}{RGB}{255,255,255}  % white
}

\lstdefinelanguage[2]{protobuf}{%
    sensitive=true,%
    morecomment=[l]{//},%
    morecomment=[s]{/*}{*/},%
    morestring=[b]{"},%
    % For the keywords of Protocol Buffers
    % see https://developers.google.com/protocol-buffers/docs/proto
    morekeywords={enum,oneof,map,syntax,public,import,option,package,message,%
        group,optional,required,repeated,default,reserved,extend,extensions,%
        to,max,service,rpc,returns,true,false},%
    % Basic types
    % see https://developers.google.com/protocol-buffers/docs/proto#scalar
    morekeywords=[2]{%
        double,float,int32,int64,uint32,uint64,sint32,sint64,%
        fixed32,fixed64,sfixed32,sfixed64,bool,string,bytes},%
    % Options
    % taken from 'google/protobuf/descriptor.proto'
    morekeywords=[3]{%
        % Generic Options
        deprecated, uninterpreted_option,%
        % File Options
        java_package,java_outer_classname,java_multiple_files,%
        java_generate_equals_and_hash,java_string_check_utf8,optimize_for,%
        go_package,cc_generic_services,java_generic_services,%
        py_generic_services,cc_enable_arenas,obj_class_prefix,%
        csharp_namespace,%
        % Message Options
        message_set_wire_format,no_standard_descriptor_accessor,map_entry,%
        % Field Options
        ctype, packed,jstype,lazy,weak,%
        % Enum Options
        allow_alias}%
}
\lstalias[]{protobuf2}[2]{protobuf}

\lstdefinelanguage[3]{protobuf}[2]{protobuf}{%
    % Language keywords
    % see https://developers.google.com/protocol-buffers/docs/proto3
    deletekeywords={
        % 'group' was marked as deprecated in protobuf2; now it's disallowed
        group,%
        % in protobuf3 the Any type replaces extensions (from protobuf2)
        extensions, to, extend, max,%
        % 'required' is no longer allowed
        required,%
        % 'optional' is default; stating it explicitly is disallowed
        optional,%
        % explicit default values are no longer allowed
        default}%
}
\lstalias[]{protobuf3}[3]{protobuf}


\lstdefinestyle{protobuf}{
  frame=lines,
  xleftmargin=\parindent,
  belowcaptionskip=1\baselineskip,
  backgroundcolor=\color{proto_background},
  basicstyle=\color{proto_basic}\footnotesize\ttfamily,
	keywordstyle=[1]\color{proto_keyword},
	keywordstyle=[2]\color{proto_type},
	keywordstyle=[3]\color{proto_options},
	commentstyle=\color{proto_comment},
	stringstyle=\color{proto_string},
  numberstyle=\color{proto_number}\tiny,
  identifierstyle=\color{proto_ident},
	numbers=left,
	numbersep=5pt,
	breaklines=true,
	showstringspaces=false,
	tabsize=2,
  % This 'literate' block is responsible for colouring numbers
  % appearing in the code
  literate={0}{{\textcolor{proto_digits}{0}}}{1}%
           {1}{{\textcolor{proto_digits}{1}}}{1}%
           {2}{{\textcolor{proto_digits}{2}}}{1}%
           {3}{{\textcolor{proto_digits}{3}}}{1}%
           {4}{{\textcolor{proto_digits}{4}}}{1}%
           {5}{{\textcolor{proto_digits}{5}}}{1}%
           {6}{{\textcolor{proto_digits}{6}}}{1}%
           {7}{{\textcolor{proto_digits}{7}}}{1}%
           {8}{{\textcolor{proto_digits}{8}}}{1}%
           {9}{{\textcolor{proto_digits}{9}}}{1}%
           {.0}{{\textcolor{proto_digits}{.0}}}{2}%
           {.1}{{\textcolor{proto_digits}{.1}}}{2}%
           {.2}{{\textcolor{proto_digits}{.2}}}{2}%
           {.3}{{\textcolor{proto_digits}{.3}}}{2}%
           {.4}{{\textcolor{proto_digits}{.4}}}{2}%
           {.5}{{\textcolor{proto_digits}{.5}}}{2}%
           {.6}{{\textcolor{proto_digits}{.6}}}{2}%
           {.7}{{\textcolor{proto_digits}{.7}}}{2}%
           {.8}{{\textcolor{proto_digits}{.8}}}{2}%
           {.9}{{\textcolor{proto_digits}{.9}}}{2}%
           % We need to add some hacks - otherwise 'listings' would
           % colour (only) the digits in the types instead of the type
           {int32}{{\textcolor{proto_type}{int32}}}{5}%
           {int64}{{\textcolor{proto_type}{int64}}}{5}%
           {uint32}{{\textcolor{proto_type}{uint32}}}{6}%
           {uint64}{{\textcolor{proto_type}{uint64}}}{6}%
           {sint32}{{\textcolor{proto_type}{sint32}}}{6}%
           {sint64}{{\textcolor{proto_type}{sint64}}}{6}%
           {fixed32}{{\textcolor{proto_type}{fixed32}}}{7}%
           {fixed64}{{\textcolor{proto_type}{fixed64}}}{7}%
           {sfixed32}{{\textcolor{proto_type}{sfixed32}}}{8}%
           {sfixed64}{{\textcolor{proto_type}{sfixed64}}}{8}%
           {java_string_check_utf8}{{%
             \textcolor{proto_options}{java_string_check_utf8}}}{2}%
           {\ }{{ }}{1},
	prebreak=\raisebox{0ex}[0ex][0ex]{\ensuremath{\hookleftarrow}},
	upquote=true,
}
\SetProtoColorsBlueish{}
%end proto files

\onehalfspacing
\Huge
\begin{document}
% ---------------------------------------------------------------
\frontmatter
    \thispagestyle{empty}
\newgeometry{left=2.5cm,right=2.5cm}
\begin{center}

\vspace*{1.4cm}
{\LARGE Technical University of Berlin}

\vspace{0.5cm}
{\large Faculty V - Mechanical Engineering and Transport Systems\\[1mm]}
{\large Department of Machine Tools and Factory Management\\[1mm]}
{\large Divison of Industrial Automation\\[5mm]}

Pascalstr. 8-9\\
10587 Berlin\\
https://www.iat.tu-berlin.de\\

\vspace*{1cm}

\includegraphics[width=3.5cm]{tu_logo.jpg}

\vspace*{1.0cm}

{\LARGE Master's Thesis}\\

\vspace{1.0cm}
{\LARGE \textbf{Automatic Generation of Object Detection Services with Varying Detection Methods and Interfaces}}\\
%\vspace*{0.3cm}

{\LARGE \textbf{}}\\
\vspace*{0.5cm}
{\LARGE Nikolas Keuck}\\
388015\\
nikolas.keuck@gmail.com\\
Major: Computational Engineering Sciences
\\
\vspace*{0.5cm}
August 22, 2019\\ % 	date of submission
\vspace*{0.5cm}

Referee: Prof. Dr.-Ing. Jörg Krüger\\
Tutor: Dipl.-Ing. Martin Rudorfer

%\vspace{2cm}


\end{center}
\restoregeometry

   	\thispagestyle{empty}
    \cleardoublepage
    
    \thispagestyle{empty}
\vspace*{1.0cm}

\begin{center}
    {\LARGE \textbf{Acknowledgments}}
\end{center}

\vspace*{0.5cm}

First of all, I would like to thank my supervisor Martin Rudorfer for his constant flow of helpful ideas, high responsiveness and willingness to take time to support me.. It sums up to 15 hours of face to face meetings, 1.5 hours of video calls, 42 e-mails and unknown time for proofreading.\\ \\
\noindent
To my colleagues, namely Wilma, Marian and Nils: Thank you for being so cooperative. You made working full time and writing a master's thesis possible!\\ \\
\noindent
Last but not least, I would like to thank my family and friends who accompanied me over the years of my studies:\\ \\

\textbf{Thank you!}
    \thispagestyle{empty}
    \cleardoublepage
    
    \include{./misc/self-assertion} 
    \thispagestyle{empty}
    \cleardoublepage
    
    
    \thispagestyle{empty}
\vspace*{1.0cm}

\begin{center}
    \textbf{Abstract}
\end{center}

\vspace*{0.5cm}

\noindent
Contemporary manufacturing relies on monolithic object detection (OD) software architecture to achieve high-performance automation. As production machines typically have a life span of over 20 years, object detection aligns to this rigid pattern. Typically, necessary OD hardware is delivered along with the software. In the last decade, OD research leaped forward while product life cycles shortened concurrently; thus it is desirable to get more frequent updates. Subsequently, the goals of this thesis are eliminating hardware and saving maintenance costs while simultaneously keeping up to date with state of the art OD methods (ODM). I introduce \textit{Recipe Generator}, a framework based on service-oriented architectures (SOA). It trains  OD services (ODS) stored in a Docker Hub with a Computer Aided Design (CAD) file. Then, it combines pairs of trained ODS and camera image acquisition services to recipes. Recipes are transferred, prepared and executed via an Open Platform Communication Unified Architecture (OPC UA) Vision Server for easy manufacturing integration. The services are dockerized and communicate via Google Remote Procedure Calls (gRPC) or Representational State Transfer (REST). The main advantages of this approach are the flexible reuse of existing ODMs and ease to add new ones in a preferred language and platform. The main challenges are to increase the reliability and availability of the framework by, e.g., an Enterprise Service Bus (ESB).
    \thispagestyle{empty}
    \cleardoublepage
    \thispagestyle{empty}
\vspace*{1.0cm}

\begin{center}
    \textbf{Zusammenfassung}
\end{center}

\vspace*{0.5cm}

\noindent

Die moderne Fertigung setzt auf monolithische Softwarearchitekturen in der  Objekterkennung (OE), um eine leistungsstarke Automatisierung zu erreichen. Da Produktionsmaschinen in der Regel eine Lebensdauer von über 20 Jahren haben, richtet sich die Objekterkennung nach diesem starren Muster. In der Regel wird die erforderliche OE-Hardware zusammen mit der Software geliefert. In den letzten zehn Jahren hat die OE-Forschung einen Sprung nach vorne gemacht, während sich gleichzeitig die Produktlebenszyklen verkürzt haben. Daher ist es wünschenswert, häufigere Aktualisierungen zu erhalten. Ziel dieser Arbeit ist es, Hardware zu eliminieren, Wartungskosten zu senken und gleichzeitig auf dem neuesten Stand der OE-Methoden zu bleiben. Diese Arbeit stellt \textit{Recepy Generator} vor, eine Rahmenstruktur (Framework), das auf Dienste-orientierten Architekturen basiert. Es trainiert OE-Services, die in einem Docker-Hub gespeichert sind, mit einer computergestützten Designdatei (CAD). Anschließend werden Paare von trainierten OE-Diensten und Bilderfassungsdiensten zu Rezepten kombiniert. Die Rezepte werden über einen OPC UA Vision-Server in die Fertigung übertragen und dort vorbereitet und ausgeführt. Die Dienste sind dockerisiert und kommunizieren über Googles Fernprozeduraufrufe (gRPC) oder REST. Die Hauptvorteile dieses Ansatzes sind die flexible Wiederverwendung vorhandener OE-Methoden und das einfache Hinzufügen neuer Methoden in einer bevorzugten Sprache und Plattform. Es gibt zwei Hauptherausforderungen. Die erste besteht darin, die Leistung, Zuverlässigkeit und Verfügbarkeit des Frameworks durch z.B. zeitsensitive Vernetzung (TSN) zu erhöhen. Die zweite besteht darin, das Framework mit einem Katalog für sofort betriebsbereite Dienste zu erweitern.
    \thispagestyle{empty}
    \cleardoublepage
    
    % \setcounter{tocdepth}{2}
    \tableofcontents
    % \thispagestyle{empty}
    
    \listoffigures
    % \thispagestyle{empty}
    
    \listoftables
    % \thispagestyle{empty}
    
    \chapter*{List of Acronyms}
\thispagestyle{empty}
\begin{tabbing}
spacespacespace \= space \kill
API	 \> 	Application Programming Interface	 \\
ARM	 \> Acorn RISC Machine\\
CPU	 \> 	Central Processing Unit	 \\
CRUD	 \> 	Create, Retrieve, Update, Delete	 \\
FSC	 \> 	Future Smart Cam	 \\
GUI	\>	Graphical User Interface \\
HTML	\>	Hypertext Markup Language \\
HTTP	 \> 	Hypertext Transfer Protocol	 \\
ISO    \>  International Organization for Standardization\\
JSON	\>	JavaScript Object Notation \\
OSI     \>  Open Systems Interconnection\\
PK	 \> 	Primary Key	 \\
RAM    \> Read Access Memory \\
REST    \> Representational State Transfer \\
SOAP	 \> 	Simple Object Access Protocol	 \\
URL	 \> 	Uniform Resource Locator	 \\
WLAN	 \> 	Wireless Local Area Network	 \\
\end{tabbing}
\endinput

    % \thispagestyle{empty}
    
    \chapter*{Important Terms in this Thesis}
\thispagestyle{empty}

\extrarowsep=0.6cm\begin{tabu} to \linewidth{lX}
    Interface & Enables communication between multiple applications (inter-process communication) on the same or different computer. Examples are DCOM, RPC and REST. So interface in the context of this thesis means data-oriented instead of functional communication module. The latter is called interface \textit{semantic}.\\ 
    Object Detection System & A process of camera image acquisition, detecting an object inside that camera image and returning its pose.\\
    Recipe &  	Describes properties, procedures and parameters for a machine vision job. \cite{VDMA2018OPCSpecification}	 \\ 
    Service & A mechanism to enable access to one or more capabilities, where the access is provided using a prescribed interface and is exercised consistent with constraints and policies as specified by the service description. \cite{MacKenzie2006ReferenceStandard}
\end{tabu}
\endinput
    % \thispagestyle{empty}

    
    
% --------------------------------------------------------------

\mainmatter % comment single chapters for faster compilation
\parskip=2ex\relax
    \chapter{Introduction\label{cha:chapter1}}
There is a huge variety of different object detection algorithms reaching from feature-based to template-based methods, and more recently also Deep Learning based approaches. Especially the latter have conquered the field very rapidly and have consistently improved the state of the art in many established object detection benchmarks. However, this improvement has not yet arrived at the shop floors of the manufacturing companies. Some of the reasons are directly connected to the method, such as the large amounts of required training data. However, another reason is that the current machine vision infrastructure is not flexible enough to keep track with the dynamic development in the field of object detection. To deploy or even only try out a new method typically requires a certain invest in cost and time that has to be justified very well. This work flow can be greatly accelerated by using Service-Oriented Architectures (SOA).

The goal of this thesis is to simplify the process of programming an object detector and integrating it into the manufacturing process. In fact, the great aim is to not program anything at all – instead, a framework shall be proposed that allows to generate an object detection service from a single example image.
The service should detect the object with a specified method and should automatically have the appropriate interfaces for convenient integration into SOA. The great advantage is that new methods can be tested and deployed without any effort, as we can simply generate and deploy a service which has the same interface as our old one but an updated method.
 
    \chapter{State of the Art\label{cha:chapter2}}
\section{Object Detection Methods}
State of the art object detection methods include \textit{scale invariant feature transformation} (SIFT) \cite{Lowe2004DistinctiveKeypoints}, \textit{speeded up robust features} (SURF) \cite{Bay2008Speeded-UpSURF}, \textit{binary robust invariant scalable
keypoints} (BRISK) \cite{Leutenegger2011BRISK:Keypoints}, \textit{oriented fast and rotated BRIEF} (ORB) \cite{Rublee2011ORB:SURF}, \textit{Accelerated KAZE} (AKAZE, KAZE meaning wind in Japanese) \cite{Alcantarilla2012KAZEFeatures, Alcantarilla2013FastSpaces} and \textit{template matching} (TM) \cite{Brunelli2009TemplatePractice}. The two latter shall be discussed here briefly.

AKAZE feature matching is a pyramidal object detection algorithm. The Pyramidal approach declines the image ratio with every filtering step. Spanning over several scales features are detected and described. In contrast to other established algorithms like SIFT, which use a linear scale space for filtering, AKAZE instead uses a nonlinear scale space with the recently developed numerical model \textit{Fast Explicit Diffusion}. The advantage of the nonlinear scale space is depicted in figure  \ref{skalenraum}. Prominent features like edges and corners are preserved. 
\begin{figure}[ht]
	\centering
  \includegraphics[width=\textwidth]{nonlinearscalespace.png}
	\caption[Gauss filtering]{Top row: Gauss filtering with linear scale space and increasing standard deviation. Bottom Row: Non linear diffusion scale space. \cite{Alcantarilla2012KAZEFeatures}}
	\label{skalenraum}
\end{figure}
The outcome of AKAZE is depicted in fig \ref{AKAZE}. Matches are shown as dots connected with lines between the two pictures of the same object from another point of view.
\begin{figure}[ht]
	\centering
  \includegraphics[width=\textwidth]{AKAZE.png}
	\caption[AKAZE Feature Matching]{AKAZE Feature Matching for a grafitti viewed from two different angles. Best viewed in color. \cite{Documentation.LastVisited2018-11-15.TutorialMatching}}
	\label{AKAZE}
\end{figure}
TM unlike AKAZE matches templates instead of features. Fig \ref{templatematching} illustrates this process. In the left image the face of the man is to be found. The template is the little cut-out in the middle. Pixel by pixel the template is being convoluted with the original image and rated with a metric. The resulting resolution matrix is depicted on the right. Bright areas indicate potential findings. At the brightest point the template is rightfully suggested.

\begin{figure}[ht]
	\centering
  \includegraphics[width=\textwidth]{templatematching.png}
	\caption[Template Matching]{In the picture on the left the template in the middle is to be found. Depicted on the right is the resolution matrix with potential findings indicated with the bright color and the found area of the template in the original image.\cite{Documentation.LastVisited2018-11-15.2014TemplateMatching}}
	\label{templatematching}
\end{figure}

\section {Service Interfaces}
\label{serviceinterfaces}
In this section potential client-server-based interfaces and underlying protocols for the object detection service shall be discussed. The evaluated interfaces are \textit{Advanced Message Queuing Protocol} (AMQP), \textit{message queuing telemetry transport} (MQTT), \textit{representational state transfer} (REST), \textit{Google remote procedure calls} (gRPC), \textit{graph query language} (GraphQL) and \textit{open platform communication unified architecture vision} (OPC UA Vision). 

Both AMQP 0.x and MQTT are broker based protocols. They are protocols specialized for machine-to-machine (M2M) communication. Clients can be sensors, programmable logic controllers etc., the server is a broker connecting the clients. A broker is a central instance mediating between parties. Clients can subscribe to various message queues called topics. Telemetry data can then be published and read from these topics handled by the broker. The clients dynamically change between publisher and subscriber. Fig \ref{MQTT} illustrates the MQTT architecture.  \cite{Banks2014MQTT3.1.1} RabbitMQ is a message broker software which supports AMQP 0.x natively and MQTT via a plugin. \cite{Lastvisited2018-15-122018WhichSupport} AMQP needs to be distinguished between 0.x and 1.0, as the underlying messaging paradigm has been completely revised. While for version 0.x strict publishing/subcription messaging is required, version 1.0 is based on a peer-to-peer connection where a broker is not required, although possible. Due to the more complex version of version 1.0, fewer implementations exist. \cite{Dizdarevic2018SurveyIntegration}

\begin{figure}[ht]
	\centering
  \includegraphics[width=0.9\textwidth]{MQTT.png}
	\caption[MQTT Architecture]{MQTT Architecture with the broker in the middle as a mediator between the clients.\cite{2018-11-24Pure-javascript-MQTT-broker}}
	\label{MQTT}
\end{figure}

REST is an architectural paradigm describing how distributed systems can communicate with each other. It consists of five mandatory and one optional restriction. If any of the five mandatory restrictions are violated, an architecture cannot be RESTful. The restrictions are 
\begin{enumerate}
    \item Client–server architecture
    \item Statelessness
    \item Cacheability
    \item Layered system
    \item Uniform interface
    \item Code on demand (optional)
\end{enumerate}
REST was developed by Roy Fielding alongside to HTTP/1.1 and although is not dependent on it, is the primary used protocol to implement REST. Thus, many web pages fulfill these restrictions naturally. REST messages are usually human readable JSON files. Unlike MQTT, REST does not rely on a broker.  \cite{Fielding2000ArchitecturalArchitectures} 

For many cases in the past it was hard for the maintainers to adhere to all REST principles due to its strict nature. Moreover, REST is usually implemented with HTTP/1.1. In 2015 HTTP/2 was released to adress the flaws of its predecessor. \cite{SayfanLastvisited2018-11-242018RESTAPIs} Among those are the lack of ability of constant data streaming, latency issues etc. gRPC fully takes advantage of HTTP/2 and thus has some advantage over REST. It uses stubs to describe an interface. Stubs are independent of any programming language. Unlike in most implementations of REST, gRPC does not use textual transport data like JSON but relies on Protobuf, a binary buffer. Moreover, Google offers an \textit{Extensible Server Proxy} (ESP), which offers a transcoding from HTTP/JSON to gRPC. \cite{2018-11-242017Grpc-gateway, Lastvisited2018-11-272018CloudGRPC} 

\begin{figure}[ht]
	\centering
  \includegraphics[width=0.7\textwidth]{gRPC_ESP.png}
	\caption{Deployed endpoints gRPC application. \cite{Lastvisited2018-11-272018CloudGRPC}}
	\label{ESP}
\end{figure}

GraphQL is a data query and -manipulation language. It was developed by Facebook and is open source since 2015. Compared to REST it is has a more flexible and efficient approach. Increase in efficiency over REST is based on faster mobile data access, and more flexibility for the API to let clients access precisely the data it needs,  i.e. the server modifies the data with respect to the clients needs instead of providing one rigid resource. \cite{LastVisited2018-15-122018BasicsIntroduction} 

OPC UA Vision is a companion specification adressing robotics and object detection. \cite{VDMA2018OPCSpecification} It's scope is to standardize the interfaces between the machine vision system and its process environments. A possible use case would be a conveyor belt that should be halted if any higher system level tells it to do so. Then, image acquisition can conducted by the machine vision system. As soon as image acquisition is done, an event should be fired which tells the higher production system the result of the acquisition. The conveyor belt can then continue. In June 2018 the first part of OPC UA Vision was released. This parts includes the basic skills on the infrastructure layer, such as result transmission, machine status etc. The following parts which have not been released yet should include machine vision skills such as post detection or a completeness check. As of now, there is no implementation of OPC UA Vision available and thus can only be used on an conceptual level.

\section {Deployment Options}
\label{deploymentoptions}
In the last decade, most applications had a monolithic character which did not focus much on scalability and interfaces. With the progress in digitalization, applications had to become more flexible and faster. To address this problem, the container architecture came about. Unlike a virtual machine which needs an operating system, runtime, system variables etc. to operate well, docker and others are sandbox systems which can imply all the mentioned features and furthermore can run on almost any operating system. See figure \ref{container} for an illustration. In the following, two possible deployment options are discussed, namely Heroku and Docker.\cite{WurbsLastvisited2018-11-272017DockerVeraendern}

\begin{figure}[ht]
	\centering
  \includegraphics[width=0.7\textwidth]{containervsvm.jpg}
	\caption[Docker vs Virtual Machine Architecture]{Virtual machine architecture on the left versus container architecture on the right. Docker does not rely on a hypervisor.\cite{WurbsLastvisited2018-11-272017DockerVeraendern}}
	\label{container}
\end{figure}

Docker is an open source standard for operating-system-level virtualization. If docker is installed on an operating system, it is possible to run several apps on the machine simultaniously, with low start and stop times and little overhead. In combination with a continuous integration and continuous delivery platform, development and operations can be harmonized.

Heroku is a platform as a service provider which underlying technology shares some core concepts with Docker. E.g. BuildPacks are a set of scripts which are used to setup the final state of an image. The pendant on the Docker side is called Dockerfile. See \cite{ThurigLastvisited2018-27-112014DockerHeroku} for a full description of the similarities and \ref{dockerandheroku} for a list of pendants. However, there are also differences for the two alternatives. The main one is the dependency on the Heroku platform on the Heroku side, whereby on the Docker side one is completely flexible to choose any environment from Raspberry Pi to cloud platform providers like Amazon Web Services. This also means a surplus of workload on infrastructure on the Docker side. Also, one is less flexible on the prices. Heroku has a staged price model ranging from 0 to 500\,\$ per month and dyno. Docker is again more flexible in letting one just paying for the hosting and storaging and leaving the additional features provided by Heroku aside. \cite{ChrisLastvisited2018-11-272017WhyDocker} 


\begin{table}
\begin{center}
      \caption{Similar core concepts of Docker and Heroku. \cite{ThurigLastvisited2018-27-112014DockerHeroku}}
  \begin{tabular}{ l | l }
    Docker & Heroku  \\ \hline
Dockerfile &	BuildPack \\ 
Image	& Slug\\ 
Container&	Dyno\\ 
Index	&Add-Ons\\ 
CLI	&CLI
  \end{tabular}
  \label{dockerandheroku}
\end{center}
\end{table}

\section{Service Oriented Architectures}
Industrial Image Processing Applications as Orchestration of Web Services
\section{Communication between Object Detection System and Client}
If two humans want to communicate with one another, they need matching channels and they need to speak the same language. A channel is a mean of transport for information, e.g. sign language, smoke signs, mobile phones etc. If one entity tries to call someone via phone if the other does not have a phone, the caller enters the wrong number, the other has its phone turned off, they cannot communicate. In case they both have a phone, the called entity answers and both speak a common language (e.g. English), exchange of information can be achieved. Communication within technical systems faces the same challenges. As for the right channel, models like the OSI basic reference model for information technology standardized by ISO layer the transfer of information from the physical layer consisting of peaks in currents and voltages up until the application layer which includes direct user interaction, resource availability and so forth. \cite{InternationalOrganizationForStandardization1996ISO/IECEd.} This model and the protocols adhering to it ensure that information is delivered safely between communicating entities. However, this model does not imply the semantics of the payload or the language as stated in the analogy above. A currently proposed semantic standard for OD processes is \textbf{OPC UA Vision}. It includes a finite state machine abstracting an industrial system from its diverse conditions and transitions. Moreover it offers an information model covering the administration of recipes, configurations and results. With the help of the state machine it is defined which information of the information model is retrievable. The content of the three administration objects remain proprietary with the advantage of covering a broad range of OD scenarios.

According to the specification, powering up and shutting down a vision system are mandatory processes and thus should be handled in a standardized manner. Also, the handling of errors should be the same for all vision systems. The design of the core operation state however shall remain with the manufacturer. Automatic mode as sub-state of operational mode as   is one proposed way of designing it. It reflects the goal of specifying a system to be easily integrated into automated production and inspection systems. An example for this operation would be a PLC guiding an inspection system for position determination. The state machine for a typical vision system is depicted in \ref{fig:OPCStateMachineAutomatic}. When powered on, the system enters the preoperational state through loading a configuration marked as active. From there an operation mode is either automatically chosen by the system or manually triggered. An operation mode is any sub-state machine of the operational state. The automatic mode is chosen and enters the initial state. Then a recipe can be prepared, describing properties, procedures and parameters for a machine vision job. The recipe may include information for a single and / or continuous execution. A single execution would be e.g. determining the pose of an object, a continuous execution could be monitoring and surveillance systems which constantly process and acquire data. When the system is done with an execution or execution step, e.g. taking a picture from one of four angles, it sends results asynchronously to the client. If the system is shut down, it should be put into the halt mode first where a safe power off is assured. From all states it is possible to enter the error state. Errors are handled aligning with their severity and sometimes need acknowledgement or confirm by a human, before the system can be reset to preoperational state.

\begin{figure}[ht]
    \centering
    \includegraphics[width=\textwidth]{img/OPCUAVisionVisionAutomaticModeStateMachineStates.pdf}
    \caption[OPC UA Vision state machine in automatic operation mode]{OPC UA Vision state machine in automatic operation mode. Red Lines indicate automatic transitions induced by the vision system with optional effects prefixed with a slash. Black lines indicate method induced transitions with the method name as trigger. The black circle is the entry point of the state machine. All of the states can have optional sub-state machines. States marked in grey are substates.\cite{VDMA2018OPCSpecification}}
    \label{fig:OPCStateMachineAutomatic}
\end{figure}

The information model formally describes all datasets, types, methods, address- and namespaces. See fig. \ref{fig:OPCInfoModelOverview} for an overview and \ref{fig:OPCInfoModelNotation} for an explanation of the notation. The StartSingleJob method (not depicted in the overview) for example triggers transisition from state Ready to SingleExecution in \ref{fig:OPCStateMachineAutomatic}. Its signature consists of following parameters:

\begin{tabbing}
    space \= space \= spacespacespace \= spacespacespacespace \= spacespacespace \kill
    \>  StartSingleJob(\\
    \>  \>  (in)	 \> 	String          \> MeasId\\
    \>  \>  (in)	 \> 	String          \> PartId\\
    \>  \>  (in)	 \> 	RecipeIdType    \> RecipeId\\
    \>  \>  (in)	 \> 	ProductIdType   \> ProductId\\
    \>  \>  (out)	 \> 	String          \> JobId\\
    \>  \>  (out)	 \> 	Int32           \> Error); 
\end{tabbing}

\begin{figure}
    \centering
    \includegraphics[height=0.9\textheight]{img/OPCUAVisionInformationModelOverview.pdf}
    \caption[OPC UA Vision Information Model Overview]{OPC UA Vision Information Model Overview. See fig. \ref{fig:OPCInfoModelNotation} for a description of the notation.\cite{VDMA2018OPCSpecification}}
    \label{fig:OPCInfoModelOverview}
\end{figure}

\begin{figure}[ht]
    \centering
    \includegraphics[width=0.8\textwidth]{img/OPCUAVisionInformationModelNotation.pdf}
    \caption[OPC UA Vision Information Model Notation]{OPC UA Vision Information Model Notation.\cite{VDMA2018OPCSpecification}}
    \label{fig:OPCInfoModelNotation}
\end{figure}

The generated OD services should adhere to the OPC UA Vision companion specification. 

To enable exchange with OPC UA servers without the use of an OPC UA stack (enabling
greater interoperability with OPC UA), a holistic translation must translate both structural
and foundational aspects of OPC UA. This means that translation must bypass OPC UA
transport protocols, secure channel management, serialization and OPC UA services - all
in a way that avoids semantic dependencies and preserves translation transparency.

OPC echtzeit praktisch! TSN
\section{Service Generation}
\subsection{Object Detection Method Integration}

\begin{figure}[ht]
    \centering
    \includegraphics[width=0.8\textwidth]{img/ServiceArchitecture.pdf}
    \caption{Service Interface Architecture.}
    \label{fig:ServiceInterArchit}
\end{figure}

\begin{figure}[ht]
    \centering
    \includegraphics[width=\textwidth]{img/ServiceGenerationExample.pdf}
    \caption{Service generation example in industrial context.}
    \label{fig:ServiceGenExa}
\end{figure}

\subsection{Interface Definition and Protocol Adapters}
For client usability reasons it is beneficial if a service is addressable via several protocols. For example a service using GRPC based on http/2 might want to interact with older web clients which use REST based on http/1.1. IoT devices sometimes only support lightweight publish-subscribe protocols such as MQTT. 

This is also the main issue about gateway middleware, as established in the related
works-section. Because they lack transparency, there is a heavy leakage of endpoint semantics present, which often forces the consumer to understand both the gateway interface
and the endpoint interface. As this creates a semantic dependency, interoperability solutions such as gateway middleware only tend to move the problem rather than solve it.

\begin{table}
    \centering
    \begin{tabular}{|c|cccc|}
    \hline
         \diagbox[]{from}{to} & OPC UA & GRPC & REST & MQTT\\\hline
         OPC UA &  \\
         GRPC & \\\hline
         
    \end{tabular}
    \caption{MT translation}
    \label{tab:protoAdptr}
\end{table}

\section{Interoparibility Layers}
siehe Integration of OPC UA with IIoT Protocols chapter 3.1.3

\section{Evaluation Parameters}
As opposed to monolithic applications SOAs are highly agile and easily integrated into distributed systems. This hypothesis has to be supported by qualitative and quantitative test results. An example for a qualitative test would be a test center where the same OD algorithm has to be integrated into a SOA and a monolithic application. The integration should be executed by several equally skilled groups of which one half start with the SOA and the other with the monolithic application. 
Quelle mit Evaluierfragen heranziehen.
Quantitative parameters could be the size of the service containers, the CPU and RAM load they utilize. The usability of the service is also of high impartance, e.g. the number of protocols that the interface offers, the documentation of the interface etc. 


    % \chapter{Concept\label{cha:chapter3}}
\section{Communication between Object Detection System and Client}
If two humans want to communicate with one another, they need matching channels and they need to speak the same language. A channel is a mean of transport for information, e.g. sign language, smoke signs, mobile phones etc. If one entity tries to call someone via phone if the other does not have a phone, the caller enters the wrong number, the other has its phone turned off, they cannot communicate. In case they both have a phone, the called entity answers and both speak a common language (e.g. English), exchange of information can be achieved. Communication within technical systems faces the same challenges. As for the right channel, models like the OSI basic reference model for information technology standardized by ISO layer the transfer of information from the physical layer consisting of peaks in currents and voltages up until the application layer which includes direct user interaction, resource availability and so forth. \cite{InternationalOrganizationForStandardization1996ISO/IECEd.} This model and the protocols adhering to it ensure that information is delivered safely between communicating entities. However, this model does not imply the semantics of the payload or the language as stated in the analogy above. A currently proposed semantic standard for OD processes is \textbf{OPC UA Vision}. It includes a finite state machine abstracting an industrial system from its diverse conditions and transitions. Moreover it offers an information model covering the administration of recipes, configurations and results. With the help of the state machine it is defined which information of the information model is retrievable. The content of the three administration objects remain proprietary with the advantage of covering a broad range of OD scenarios.\\

According to the specification, powering up and shutting down a vision system are mandatory processes and thus should be handled in a standardized manner. Also, the handling of errors should be the same for all vision systems. The design of the core operation state however shall remain with the manufacturer. Automatic mode as sub-state of operational mode as   is one proposed way of designing it. It reflects the goal of specifying a system to be easily integrated into automated production and inspection systems. An example for this operation would be a PLC guiding an inspection system for position determination. The state machine for a typical vision system is depicted in \ref{fig:OPCStateMachineAutomatic}. When powered on, the system enters the preoperational state through loading a configuration marked as active. From there an operation mode is either automatically chosen by the system or manually triggered. An operation mode is any sub-state machine of the operational state. The automatic mode is chosen and enters the initial state. Then a recipe can be prepared, describing properties, procedures and parameters for a machine vision job. The recipe may include information for a single and / or continuous execution. A single execution would be e.g. determining the pose of an object, a continuous execution could be monitoring and surveillance systems which constantly process and acquire data. When the system is done with an execution or execution step, e.g. taking a picture from one of four angles, it sends results asynchronously to the client. If the system is shut down, it should be put into the halt mode first where a safe power off is assured. From all states it is possible to enter the error state. Errors are handled aligning with their severity and sometimes need acknowledgement or confirm by a human, before the system can be reset to preoperational state.\\

\begin{figure}[ht]
    \centering
    \includegraphics[width=\textwidth]{img/OPCUAVisionVisionAutomaticModeStateMachineStates.pdf}
    \caption[OPC UA Vision state machine in automatic operation mode]{OPC UA Vision state machine in automatic operation mode. Red Lines indicate automatic transitions induced by the vision system with optional effects prefixed with a slash. Black lines indicate method induced transitions with the method name as trigger. The black circle is the entry point of the state machine. All of the states can have optional sub-state machines. States marked in grey are substates.\cite{VDMA2018OPCSpecification}}
    \label{fig:OPCStateMachineAutomatic}
\end{figure}

The information model formally describes all datasets, types, methods, address- and namespaces. See fig. \ref{fig:OPCInfoModelOverview} for an overview and \ref{fig:OPCInfoModelNotation} for an explanation of the notation. The StartSingleJob method (not depicted in the overview) for example triggers transisition from state Ready to SingleExecution in \ref{fig:OPCStateMachineAutomatic}. Its signature consists of following parameters:

\begin{tabbing}
    space \= space \= spacespacespace \= spacespacespacespace \= spacespacespace \kill
    \>  StartSingleJob(\\
    \>  \>  (in)	 \> 	String          \> MeasId\\
    \>  \>  (in)	 \> 	String          \> PartId\\
    \>  \>  (in)	 \> 	RecipeIdType    \> RecipeId\\
    \>  \>  (in)	 \> 	ProductIdType   \> ProductId\\
    \>  \>  (out)	 \> 	String          \> JobId\\
    \>  \>  (out)	 \> 	Int32           \> Error); 
\end{tabbing}

\begin{figure}
    \centering
    \includegraphics[height=0.9\textheight]{img/OPCUAVisionInformationModelOverview.pdf}
    \caption[OPC UA Vision Information Model Overview]{OPC UA Vision Information Model Overview. See fig. \ref{fig:OPCInfoModelNotation} for a description of the notation.\cite{VDMA2018OPCSpecification}}
    \label{fig:OPCInfoModelOverview}
\end{figure}

\begin{figure}[ht]
    \centering
    \includegraphics[width=0.8\textwidth]{img/OPCUAVisionInformationModelNotation.pdf}
    \caption[OPC UA Vision Information Model Notation]{OPC UA Vision Information Model Notation.\cite{VDMA2018OPCSpecification}}
    \label{fig:OPCInfoModelNotation}
\end{figure}

The generated OD services should adhere to the OPC UA Vision companion specification. 
\section{Service Generation}
\subsection{Object Detection Method Integration}

\begin{figure}[ht]
    \centering
    \includegraphics[width=0.8\textwidth]{img/ServiceArchitecture.pdf}
    \caption{Service Interface Architecture.}
    \label{fig:ServiceInterArchit}
\end{figure}

\begin{figure}[ht]
    \centering
    \includegraphics[width=0.8\textwidth]{img/ServiceGenerationExample.pdf}
    \caption{Service generation example in industrial context.}
    \label{fig:ServiceGenExa}
\end{figure}

\subsection{Interface Definition and Protocol Adapters}
For client usability reasons it is beneficial if a service is addressable via several protocols. For example a service using GRPC based on http/2 might want to interact with older web clients which use REST based on http/1.1. IoT devices sometimes only support lightweight publish-subscribe protocols such as MQTT. 
\section{Evaluation Parameters}
As opposed to monolithic applications SOAs are highly agile and easily integrated into distributed systems. This hypothesis has to be supported by qualitative and quantitative test results. An example for a qualitative test would be a test center where the same OD algorithm has to be integrated into a SOA and a monolithic application. The integration should be executed by several equally skilled groups of which one half start with the SOA and the other with the monolithic application. 
Quelle mit Evaluierfragen heranziehen.
Quantitative parameters could be the size of the service containers, the CPU and RAM load they utilize. The usability of the service is also of high impartance, e.g. the number of protocols that the interface offers, the documentation of the interface etc. 
    \chapter{Implementation\label{cha:chapter4}}
A proof of concept implementation of the concept introduced in \ref{cha:chapter3} is documented in this chapter. The code is available here: \ https://github.com/gurukeuck/Master-s-Thesis

\section{Virtualization Technology: Docker}
Realizing a SOA calls for means of decoupling the components and efficient tooling. Docker is the technology which was used here due to following reasons:
\subsection{Dependencies of components are handled smoothly} 
Every docker image can pull the packages and system variables it needs as specified in the dockerfile. E.g. one detector may depend on openCV 2.7.1 while another detector depends on 1.8. Both docker images that are built using their respective dockerfile are independent of each other. If no virtualization technology would be used here, a dependency handling for the whole recipe management would be necessary. In the case of two openCV versions just a slight modification of the framework may be necessary. A more drastic example would be detectors that rely on different .NET frameworks which might not be able to coexist on a system.
\subsection{Platform Independency}
 A platform in this context means operating system, e.g. Windows or Linux. The independency is twofold. Firstly, the docker engine runs on Linux (CentOS, Debian, Fedora, Oracle Linux, RHEL, SUSE and Ubuntu) and Windows Server. In an industrial context, both platforms are present and should be supported for maximum flexibility.
\subsection{Orchestration}
 Containers can be scaled if more resources are needed, an extensive monitoring is possible and the network over which the containers communicate can be configured. 
\subsection{Docker Hub}
 Docker offers (semi)-public or private repositories. They can be used by detector or camera providers to push their docker images and for the recipe management to pull them. As base image there are preconfigured environments available. For this implementation a fully functioning python environment was used as base docker image. To prevent know-how leaking the accessibility to the images on the hub can be restricted. In this implementation a local repository was used with full accessibility to the docker images.
 \subsection{Calculations on Graphical Processing Unit}
 Some detectors need excessive calculation power. If applicable, the graphical processing unit of the bare metal server the docker engine is running on can be added. It should be kept in mind that with this technique the detector docker image is dependent not only on the docker engine but also the bare metal hardware. Thus, this option should be used with caution and only if needed.
\subsection{Reason against Docker}
The reason against using Docker is making all involved parties introduced in \ref{sec:involvedparties} use docker. Camera and detector docker image providers need to add a dockerfile for building the image and pushing it to the repository. On the other side, recipe management must be able to run the docker engine

\section{Inter-Service Communication: gRPC}
In the current implementation this static. Detectors listen on port 8000, cameras on port 8011. 
\section{Programming Language: Python}
For implenting a proof of concept of the concept created in \ref{cha:chapter3}, a programming language meeting multiple requirements is mandatory. \\

\subsection{Support for Docker}

\subsection{Support for gRPC}

\section{Object Detection Methods used}
\section{Class diagram}
\section{Package diagram}
\section{Sequence Diagram}
\section{Differences to OPC UA Vision Specification}
    \chapter{Ergebnisse\label{cha:chapter5}}
Im folgendem werden die Ergebnisse des Projekts anhand der gestellten Anforderungen aus Abb. \ref{table:anforderungen} präsentiert. Die Betrachtung erfolgt Abschnittsweise mit den Meilensteinen als Gliederungspunkten.


\section{Objekterkennung}
Von den Objekterkennungsdiensten haben wir uns mehrere in Betracht gezogen, insbesondere Google Cloud Vision, Microsoft Azure Cognitive Services und Amazon Rekognition. Erfolgreich implementiert wurden die Templates für Google Cloud Vision und Amazon Rekognition. Nachdem die Authentifications-Files in der Datenbank hinterlegt wurden ist es möglich direkt oder aus der Datenbank Bilder an den oder die Services zu schicken und man erhält die Rückgabe als JSON. Damit sind die von uns geforderten Anforderungen für diesen Bereich alle erfüllt.
Die beiden Objekterkennungsdienste liefern nur Labels der erkannten Objekte. Für die Positionen der Objekte kann als Ausblick beispielsweise das von uns gebaute Tensorflowmodell als Webservice eingerichtet werden.


\section{Kamera}
Bei der Orbbec Persee handelt es sich um eine Kamera mit ARM-Prozessor. Es traten dadurch Probleme mit der Installation von OpenCV auf, da OpenCV standardmäßig nicht für ARM-Prozessoren ausgelegt ist . Für die zukünftige Verwendung muss OpenCV installiert werden. Der Quellcode für das Bildaufnehmen befindet sich schon auskommentiert im Programm. \newline
Aufgrund des wenig vorhandenen Speicherplatzes ist es nicht möglich das Tensorflow-Modell lokal zu betreiben. Tensorflow muss als Webservice implementiert werden und über das Netzwerk angesprochen werden können.

\section{Konfigurationswerkzeug für die Kamera}
Als Framework für die Web-Anwendung wird Django mit Python 3 verwendet. Als Alternative wurde Flask betrachtet, allerdings haben wir uns aufgrund von ausgezeichneter Dokumentation, großer Community und dem ebenfalls beliebten Django REST Framework für Django entschieden.
Die Funktionen für das Senden der Bilder und Empfangen der Antwort funktioniert mit und ohne Hinterlegung der Daten in einer Datenbank. Bilder hinzufügen und auch das Aufnehmen der Bilder über die Kamera ist für den Nutzer über die Oberfläche möglich und leicht verständlich. Sämtliche Funktionen funktionieren über die grafische Oberfläche, lassen sich aber auch per REST-Befehl nutzen. Services können hinzugefügt werden, dafür existieren zwei Templates für Google und für Amazon. Weiterhin besteht die Möglichkeit, andere oder eigene Services durch eine URL, einem Payload mit zusätzlichen Parametern und einem Authentifications-File zu ergänzen.
Die Antworten werden getrennt betrachtet und nicht ausgewertet.

\section{Tests}
Sämtliche integrierte Funktionen wurden manuell überprüft und auf Fehler getestet. Ein automatisiertes Testen mithilfe von Django unittests kam nicht zustande. Durch die integrierte Validierung werden großteilig fehlerhafte und ungewollte Eingaben durch den Nutzer verhindert und er wird auf den Fehler hingewiesen.
    \chapter{Fazit und Ausblick\label{cha:chapter6}}




    \chapter{Conclusion\label{cha:chapter7}}

In this thesis, I tried to generate ODSs with varying detection methods and interfaces automatically. The title divides into three challenges:
\begin{itemize}
    \item Generating ODS automatically
    \item Handling ODS with varying detection methods
    \item Handling ODS with varying interfaces
\end{itemize}
In response, I created a SOA concept which tackles these challenges. Generating ODS (semi-)automatically has been undergird with a supporting framework. It helps to combine the necessary components of camera image acquisition, ODMs and returning the pose in a standardized way. These services must adhere to a versioned semantic Train and Test definition, which is generic enough to comprehend a broad set of ODMs. Expanding the semantic to a new version is of little effort.\\
The service interfaces are currently limited to gRPC or REST. Albeit this covers two established interfaces, they lack certain functionality such as message queuing and real-time functionalities. A more holistic implementation is desirable for the future, e.g. with OPC UA over TSN. This would allow for even greater interoperability and reliability.

Further works include fulfilling all OPC UA Vision specification aspects once they are fully released. Especially the drill down to component level and a specification of how to interface cameras will bring about the possibility for high-speed implementation of new components.

Also, the framework could be upgraded with smart aspects such as choosing the right camera for a detection task or automated training when a new product type is produced. With the help of these features, recipes could be generated automatically.

% ---------------------------------------------------------------
%\backmatter % no page numbering from here
		
		% if you want to provide a glossary with explanations of important terms put it in here
    
    \bibliographystyle{unsrtnat} %vancouver
    \bibliography{./bib/references}

% \backmatter
    
\appendix
% \begin{appendices}

\chapter{Roles}
\label{sec:involvedparties}
\pagenumbering{roman}
\extrarowsep=0.6cm\begin{longtabu} to \linewidth{lX}
    Recipe Generator Client & The instance calling recipe generator methods Train, Test, Transfer, GenRecipe and GetConfig. It needs to have a certain intelligence to decide when to call which methods with the appropriate parameters. In the current state, this will most likely be a production engineer. It is assumed that adding or altering a recipe does not happen very frequently. In later use where more business intelligence resides inside recipe generator, the client could be a component, e.g. a PLC.\\ 
    Detector Image Provider & The instance developing new ODMs. This could be research teams, community projects like OpenCV or OD specialists of the company using recipe generator. For every new ODM, it is necessary to adhere to the recipe generator API or, for a new pattern of ODMs, update the recipe generator API.\\
    Camera Image Provider &  	The instance providing new camera hardware and capable software of interacting with recipe generator.\\ 
    Framework Architect & The author or contributor of recipe generator. He or she is responsible for the SOA design, continuous development and integration and updated proto files. Often the same person as the recipe genetor client.
\end{longtabu}


\chapter{Example of a Recipe Life Cycle}
\label{chap:recipelifecycle}
To illustrate recipe management, here is a possible life cycle of a recipe taken from the OPC UA Vision specification appendix~\cite{VDMA2018OPC40100-1:2018-11}. Note that the method signatures are not necessarily exact here. 
\begin{enumerate}
    \item A recipe for ProductId-m is created externally (often centrally). 
    \item The recipe is pushed to the vision system with ExternalId-1, ProductId-m (using AddRecipe())
    \begin{itemize}
        \item It is stored there with ExternalId-1, InternalId-11.
        \item It is linked to ProductId-m on the vision system
    \end{itemize}
    \item There are further possible actions on the recipe without any particular order.
    \begin{itemize}
        \item The recipe may be edited locally later, keeping its ExternalId-1 and receiving InternalId-12.
        \item A (binary) different version of the recipe with the same ExternalId-1 may be pushed to the vision system later, receiving InternalId-13.
        \item The recipe may be linked later to ProductId-n on the vision system. Note that the external recipe 
        management does not concern itself with the InternalIds of the recipes on different vision systems. If 
        there are, due to one of these operations, several recipes on the vision system with identical 
        ExternalIds but different InternalIds, the vision system/server combination has no means of telling 
        which of these was targeted by the environment. It may choose to link all of them, or the newest one, 
        or the latest one pushed (ignoring internally edited ones). This is outside the scope of this specification.
    \end{itemize}
    \item The automation system is undergoing a change-over process to a specific product, namely ProductId-m. It 
    will re-tool the vision system
    \begin{itemize}
    \item by calling PrepareRecipe(ExternalId-1); the vision system then selects one of the existing recipes with 
    ExternalId-1 based on internal rules.
    \item by calling PrepareRecipe(ProductId-m); the vision system then selects one of the existing recipes 
    linked to ProductId-m based on internal rules. 
    \end{itemize}
    \item The vision system is commanded to process a specific product
    \begin{itemize}
        \item by calling StartJob(ExternalId-1); the vision system then starts processing with the recipe prepared for ExternalId-1. 
        \item by calling StartJob(ProductId-m); the vision system then starts processing with the recipe prepared for ProductId-m. 
        \item by calling StartJob(ExternalId-2); the vision system then throws an error because no such recipe has 
        been added or prepared.
        \item by calling StartJob(ProductId-p); the vision system then throws an error because no such recipe has 
        been added or prepared. 
    \end{itemize}
    \item If there is no error, the vision system carries out its task, going through the Executing state to return to state Ready waiting for further instructions. There are many other possibilities of errors, e.g. trying to prepare a recipe which is actually a sub-recipe, i.e., not capable of being processed by itself. 
\end{enumerate}

% \chapter{OPC UA Vision Method Signatures used in this Thesis}
\chapter{ATAM Evaluation}
\section{Quality Attribute Scenarios}
For an explanaition of the sources (i.e., roles), see annex~\ref{sec:involvedparties}.
{\renewcommand{\arraystretch}{0.7} % vertical spacing between rows
\begin{longtable}{|P{0.08\linewidth}|P{0.2\linewidth}|P{0.16\linewidth} P{0.44\linewidth}|}
\caption{Quality Attribute Scenarios}\label{tab:scen}\\
\hline
\rowcolor{Gray}
\textbf{Number} & \textbf{Quality Attribute} & \multicolumn{2}{l|}{\textbf{Scenario}}\\
\hline
\endfirsthead
\multicolumn{4}{c}%
{\tablename\ \thetable\ -- \textit{Continued}} \\
\hline
\rowcolor{Gray}
\textbf{Number} & \textbf{Quality Attribute} & \multicolumn{2}{l|}{\textbf{Scenario}}\\
\hline
\endhead
\hline \multicolumn{4}{r}{\textit{Continued on next page}} \\
\endfoot
\hline
\endlastfoot
 1 & Modifiability & Source:  & Detector Docker Image Provider\\
   & & Stimulus:  & Add a new detector\\ 
   & & Artifact:  &  Docker Registry \\ 
   & & Environment:  & Detector provider familiar with Docker, gRPC and detector proto file\\ 
   & & Response:  & New detector is added\\ 
   & & Response Measure:  & No more than five person-days of detector provider team effort is required for the implementation (legal and financial agreements are not included).\\ \hline 
 2 & Modifiability & Source:  & Camera Docker Image Provider\\
   & & Stimulus:  & Add a new camera to the vision system\\ 
   & & Artifact:  &  Camera \\ 
   & & Environment:  & Camera provider familiar with Docker, gRPC and camera proto file\\ 
   & & Response:  & New camera is added\\ 
   & & Response Measure:  & No more than four person-days of camera provider team effort is required for the implementation (legal and financial agreements are not included).\\ \hline
 3 & Modifiability & Source:  & RG Client\\
   & & Stimulus:  & Add a new proto file version\\ 
   & & Artifact:  &  Camera or Detector \\ 
   & & Environment:  & A new version of a proto file is necessary due to missing message types\\ 
   & & Response:  & New proto file is added\\ 
   & & Response Measure:  & No more than one person-day of effort is necessary. All existing camera and detector service remain functional.\\ \hline
 4 & Interoperability & Source  & Framework Architect\\
   & & Stimulus:  & Replace or scratch Docker\\ 
   & & Artifact:  & Detector and Camera Image Docker Containers\\ 
   & & Environment:  & Docker is not allowed or not possible to use\\ 
   & & Response:  & Docker is replaced or scratched\\ 
   & & Response Measure:  & No more than fifteen person-days of framework architect effort is required for the implementation\\ \hline
 5 & Performance & Source  & Detector Docker Image Provider\\
   & & Stimulus:  & Send a file to detector\\ 
   & & Artifact:  & Detector Docker Container\\ 
   & & Environment:  & CAD file input not larger than 10\,MB sent to detector\\ 
   & & Response: & File to detector sent\\ 
   & & Response Measure:  & The trip time for a request from detector client in the local network to detector during normal operation is less than 2 seconds.\\ \hline
 6 & Reliability & Source  & Framework Architect\\
   & & Stimulus  & Service failure\\ 
   & & Artifact  & All components\\ 
   & & Environment  & A sudden failure occurs in the runtime environment of a service.  \\ 
   & & Response  & After recovery, all transactions are completed or rolled back as appropriate, so the system maintains uncorrupted, persistent data.\\ 
   & & Response Measure:  & Manual test if data is corrupt or inconsistent.\\ \hline
 7 & Security & Source  & A possible attacker\\
   & & Stimulus  & A malware is included in the system\\ 
   & & Artifact  & Camera or Detector Docker Container\\ 
   & & Environment  & We have to face a denial-of-service attack.  \\ 
   & & Response  & The malware is detected quickly, best before entering the framework. Necessary information to possibly identify the attacker is logged.\\ 
   & & Response Measure:  & Our system is not operable for not more than an hour.\\ \hline
\end{longtable}
\newpage

\section{Architectural Analysis of Scenarios}
\begin{longtable}{| P{0.2\textwidth} | P{0.74\textwidth} |}
\caption{Architectural Analysis of Scenarios}\label{tab:scenan}\\
\hline
\endfirsthead
\multicolumn{2}{c}%
{\tablename\ \thetable\ -- \textit{Continued}} \\
\hline
\endhead
\hline \multicolumn{2}{r}{\textit{Continued on next page}} \\
\endfoot
\hline
\endlastfoot
\rowcolor{Gray}
\multicolumn{2}{ |l| }{\textbf{Analysis for Scenario 1}} \\ \hline
Scenario Summary & New Detector added to Docker Registry in no more than five person-days\\ \hline
Business Goal(s) & Extending the available object detection possibilities\\ \hline
Quality Attribute & Modifiability\\ \hline
Architectural Approaches and Reasoning & Detector communication depends on gRPC which covers many programming languages and a REST gateway. \newline Docker registry can be made easily accessible for the provider and is a well-known technology.\\ \hline
Risks &  Possible too much of a lock-in effect of both technologies\\ \hline
Tradeoffs &  The static Train/Test interface might not (yet) be suitable for all detection purposes, although once implemented it is very easy to use and might become more popular as a versioned standard.\\ \hline
\rowcolor{Gray}
\multicolumn{2}{ |l| }{\textbf{Analysis for Scenario 2}} \\ \hline
Scenario Summary & New camera added to vision system in no more than four person-days\\ \hline
Business Goal(s) & Possibility of heterogeneous camera ecosystem\\ \hline
Quality Attribute & Modifiability\\ \hline
Architectural Approaches and Reasoning & Docker container including or alternatively just gRPC server running on camera which allows for many programming languages \\ \hline
Risks &  The camera firmware can be unmodifiable, thus a middleware is likely to be needed. \\ \hline
Tradeoffs & In theory, the same underlying technology (Docker + gRPC) can be conveniently used for communication between recipe and camera, but in practice, at least for the camera, this might end up cumbersome without a fitting middleware. \\ \hline
\rowcolor{Gray}
\multicolumn{2}{ |l| }{\textbf{Analysis for Scenario 3}} \\ \hline
Scenario Summary & A new proto file version is added in no more than on person-day.\\ \hline
Business Goal(s) & Staying up to date with current ODMs\\ \hline
Quality Attribute & Modifiability\\ \hline
Architectural Approaches and Reasoning & gRPC offers a convenient way of expanding proto files. As long as no messages are deleted or overwritten, the system remains downward compatible.\\ \hline
Risks &  None\\ \hline
Tradeoffs &  The more messages are created, the more overhead will be sent over the channels.\\ \hline
\rowcolor{Gray}
\multicolumn{2}{ |l| }{\textbf{Analysis for Scenario 4}} \\ \hline
Scenario Summary & Replace or scratch Docker in no more than 15 person-days.\\ \hline
Business Goal(s) & Permit easy integration with new business partners. \\ \hline
Quality Attribute & Interoperability\\ \hline
Architectural Approaches and Reasoning & Detector and Camera gRPC servers/clients do not need Docker for successful communication. It offers a huge convenience for sharing them, starting them, handling dependencies, etc. A switch to e.g. Heroku would require a partial rewrite of the framework.\\ \hline
Risks &  Too much effort included rewriting the framework. Instead, an alternative framework might be used.\\ \hline
Tradeoffs & If Docker is available, the ease of service administration is very high. If not, the flexibility is higher but requires more effort.\\ \hline
\rowcolor{Gray}
\multicolumn{2}{ |l| }{\textbf{Analysis for Scenario 5}} \\ \hline
Scenario Summary & A 10\,MB file is sent to detector in less than 2 seconds. \\ \hline
Business Goal(s) & Compatible message transmission speed.\\ \hline
Quality Attribute & Performance\\ \hline
Architectural Approaches and Reasoning &  gRPC sends its data via protobufs which are a binary format specialized in high-performance serialization and deserialization. If an image has to be transferred between OPC UA Server and Client, then the emerging Time Sensitive Networking will help facing real-time transmission challenges.\\ \hline
Risks & If too slow, the process might be the bottleneck in production.\\ \hline
Tradeoffs &  None\\ \hline
\rowcolor{Gray}
\multicolumn{2}{ |l| }{\textbf{Analysis for Scenario 6}} \\ \hline
Scenario Summary & Uncorrupted, persistent data remains on the system after a service failure.\\ \hline
Business Goal(s) & Industrial-grade reliability \\ \hline
Quality Attribute & Reliability\\ \hline
Architectural Approaches and Reasoning & The framework relies on service choreography and client/server communication. Thus, every component has to ensure uncorrupted and persistent data. In the future, a messaging broker could be introduced for better reliability.\\ \hline
Risks & Currently, no reliability is granted, so a service failure will result in a failure of most components of the system \\ \hline
Tradeoffs & Adding a message broker adds reliability but also complexity. \\ \hline
\rowcolor{Gray}
\multicolumn{2}{ |l| }{\textbf{Analysis for Scenario 7}} \\ \hline
Scenario Summary & A denial-of-service attack causes no more than an hour of downtime of the framework.\\ \hline
Business Goal(s) & Offering a secure framework with little downtime. \\ \hline
Quality Attribute & Security\\ \hline
Architectural Approaches and Reasoning & The Docker Hub is intended to be accessible by many vendors to ensure a variety of detectors and cameras. Every new vendor has to ensure its integrity. New detectors and cameras have to be tested thoroughly for a possible malware function. Through the standardized interface, a malfunctioning service can best be replaced quickly by an equivalent one offered by another vendor.\\ \hline
Risks &  A service that is relied on in production reveals itself as malware can be hard to replace in the given time of one hour. Redundant services by multiple vendors can help out in this case.\\ \hline
Tradeoffs & Due to possibly high numbers of services and creative attackers, testing every new vendor and service can be a tedious task.\\ \hline
\end{longtable}
}

\chapter{Proto Files}
\label{proto}
\section{Detector}
\lstinputlisting[language=protobuf3,style=protobuf]{img/detector.proto}

\section{Camera}
\lstinputlisting[language=protobuf3,style=protobuf]{img/camera.proto}

\endinput
% \end{appendices}



    


\end{document}
