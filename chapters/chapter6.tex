\chapter{Evaluation\label{cha:chapter6}}

\section{The ATAM Method}
\epigraph{It is better to be vaguely right than exactly wrong.}{Carveth Read (1848 – 1931) philosopher and logician}
In this chapter, an excerpt of the architecture tradeoff analysis method (ATAM,~\cite{Kazman2000ATAMEvaluation}) as conducted in~\cite{Bianco2007EvaluatingArchitecture} shall be used to evaluate the current implementation and underlying concept. ATAM is a method aimed at illuminating risks in the architecture through the identification of attribute trends, rather than at precise characterizations of measurable quality attribute values.~\cite{Kazman2008ExperienceAnalysis} ATAM focuses on the analysis of quality attributes. Quality attributes, also known as \textbf{nonfunctional requirements}, include usability, performance, scalability, reliability, security and modifiability. Subsection~\ref{subseb:qualatt} provides some generic examples. Subsections~\ref{subsec:atamsteps} and~\ref{subsec:atamgoa} introduce the necessary ATAM steps and the goal of finding risks and tradeoffs. For the introduced framework it is an applicable method as it stays high level. Once it is implemented in an industrial environment, a more quantifiable method is necessary.  

ATAM has been introduced by Kazman in 1998~\cite{Kazman1998TheMethod} at Carnegie Mellon University in Pennsylvania. It was used on at least 18 architectures in the 2000s.~\cite{Bass2007RiskEvaluations} In 2013 Zalewski introduced an updated ATAM method which can be applied at an even earlier stage of architecture evaluation.~\cite{Zalewski2013BeyondSystems}

\subsection{Quality Attributes}
\label{subseb:qualatt}
The following example quality attributes are decorated with generic examples which are directly quoted from Appendix A in \textit{Evaluating a Service-Oriented Architecture}.~\cite{Bianco2007EvaluatingArchitecture} 
\subsubsection{Performance}
\begin{itemize}
    \item A sporadic request for service ‘X’ is received by the server during normal operation. The system processes the request in less than ‘Y’ seconds. 
    \item The service provider can process up to ‘X’ simultaneous requests during normal operation, keeping the response time on the server less than ‘Y’ seconds.  
    \item The roundtrip time for a request from a service user in the local network to service ‘X’ during normal operation is less than ‘Y’ seconds. 
\end{itemize}

\subsubsection{Availability}
\begin{itemize}
    \item An improperly formatted message is received by a system during normal operation. The 
system records the message and continues to operate normally without any downtime. 
    \item An unusually high number of suspect service requests are detected (denial-of-service attack), and the system is overloaded. The system logs the suspect requests, notifies the system administrators, and continues to operate normally. 
    \item Unscheduled server maintenance is required on server ‘X.’ The system remains operational in degraded mode for the duration of the maintenance. 
    \item A service request is processed according to its specification for at least 99.99\,\% of all requests. 
    \item A new service is deployed without impacting the operations of the system. 
    \item A third-party service provider is unavailable; modules that use that service respond appropriately regarding the unavailability of the external service; and the system continues to operate without failures.
\end{itemize}

\subsubsection{Security}
\begin{itemize}
    \item A third-party service with malicious code is used by the system. The third-party service is unable to access data or interfere with the operation of the system. The system notifies the system administrators. 
    \item An attack is launched attempting to access confidential customer data. The attacker is not able to break the encryption used in all the hops of the communication and where the data is persisted. The system logs the event and notifies the system administrators. 
    \item A request needs to be sent to a third-party service provider, but the provider’s identity can not be validated. The system does not make the service request and logs all relevant information. The third party is notified along with the system administrator. 
\end{itemize}

\subsubsection{Testability}
\begin{itemize}
    \item An integration tester performs integration tests on a new version of a service that provides an interface for observing output. 90\,\% path coverage is achieved within one person-week. 
\end{itemize}

\subsubsection{Interoperability}
\begin{itemize}
    \item A new business partner that uses platform ‘X’ is able to implement a service user module that works with our available services in platform ‘Y’ in two person-days.  
    \item A transaction of a legacy system running on platform ‘X’ is made available as a web service to an enterprise application that is being developed for platform ‘Y’ using the web services technology. The wrapping of the legacy operation as a service with proper security verification, transaction management, and exception handling is done in 10 person-days. 
\end{itemize}

\subsubsection{Modifiability}
\begin{itemize}
    \item A service provider changes the service implementation, but the syntax and the semantics of the interface do not change. This change does not affect the service users. 
    \item A service provider changes the interface syntax of a service that is publicly available. The old version of the service is maintained for 12 months, and existing service users are not affected within that period.  
    \item A service user is looking for a service. A suitable service is found that contains no more than ‘X’ percentage of unneeded operations, so the probability of the service provider changing is reduced. 
\end{itemize}

\subsubsection{Reliability}
\begin{itemize}
    \item A sudden failure occurs in the runtime environment of a service provider. After recovery, all transactions are completed or rolled back as appropriate, so the system maintains uncorrupted, persistent data. 
    \item A service becomes unavailable during normal operation. The system detects and restores the service within two minutes. 
\end{itemize}

\subsection{ATAM Steps}
\label{subsec:atamsteps}
ATAM consists of the following steps (directly quoted from~\cite{Bianco2007EvaluatingArchitecture}):
\begin{enumerate}
    \item Present the ATAM: The evaluation team presents a quick overview of the ATAM steps, the techniques used, and the outputs from the process. 
    \item Present the business drivers: The system manager briefly presents the business drivers and context for the architecture.
    \item Present the architecture: The architect presents an overview of the architecture.
    \item Identify architectural approaches: The evaluation team and the architect itemize the architectural approaches discovered in the previous step. 
    \item Generate the quality attribute utility tree: A small group of technically oriented stakeholders identifies, prioritizes, and refines the most important quality attribute goals in a utility tree format.
    \item Analyze the architectural approaches: The evaluation team probes the architectural approaches in light of the quality attributes to identify risks, non-risks, and tradeoffs. 
    \item Brainstorm and prioritize scenarios: A larger and more diverse group of stakeholders creates scenarios that represent their various interests. Then the group votes to prioritize the scenarios based on their relative importance. 
    \item Analyze architectural approaches: The evaluation team continues to identify risks and tradeoffs while noting the impact of each scenario on the architectural approaches. 
    \item Present results: The evaluation team recapitulates the ATAM steps, outputs, and recommendations.
\end{enumerate}

These steps are typically carried out in two phases. Phase~1 is architect-centric and concentrates on eliciting and analyzing architectural information. This phase includes a small group of technically oriented stakeholders concentrating on Steps~1 to~6. Phase~2 is stakeholder-centric, elicits points of view from a more diverse group of stakeholders, and verifies the results of the first phase. This phase involves a larger group of stakeholders, builds on the work of the first phase, and focuses on Steps~7 through~9.~\cite{Jones2001EvaluateStudy} 

\subsection{ATAM Goals}
\label{subsec:atamgoa}
It is desired to find risks and tradeoffs:
\begin{itemize}
    \item risks: architectural decisions that might create future problems for some quality attribute. A sample risk: The current version of the Database Management System is no longer supported by the vendor; therefore, no patches for security vulnerabilities will be created.
    \item tradeoffs: architectural decisions that have an effect on more than one quality attribute.     For example, the decision to introduce concurrency improves latency but increases the cost of change for the affected modules. 
\end{itemize}

\section{Architecture Evaluation of RG with ATAM Method}
In this evaluation chapter, we will cover steps 7 and 8 as described in subsection~\ref{subsec:atamsteps}. As a goal, we will identify risks and tradeoffs. The list is claimed to be incomplete and just covers some important aspects that were considered during service design. Also, in practice, ATAM can be iterated which is dropped here.

\subsection{Quality Attribute Scenarios}
For an explanaition of the sources (i.e., roles), see annex~\ref{sec:involvedparties}.
{\renewcommand{\arraystretch}{0.7} % vertical spacing between rows
\begin{longtable}{|P{0.08\linewidth}|P{0.2\linewidth}|P{0.16\linewidth} P{0.44\linewidth}|}
\caption{Quality Attribute Scenarios}\label{tab:scen}\\
\hline
\rowcolor{Gray}
\textbf{Number} & \textbf{Quality Attribute} & \multicolumn{2}{l|}{\textbf{Scenario}}\\
\hline
\endfirsthead
\multicolumn{4}{c}%
{\tablename\ \thetable\ -- \textit{Continued}} \\
\hline
\rowcolor{Gray}
\textbf{Number} & \textbf{Quality Attribute} & \multicolumn{2}{l|}{\textbf{Scenario}}\\
\hline
\endhead
\hline \multicolumn{4}{r}{\textit{Continued on next page}} \\
\endfoot
\hline
\endlastfoot
 1 & Modifiability & Source:  & Detector Docker Image Provider\\
   & & Stimulus:  & Add a new detector\\ 
   & & Artifact:  &  Docker Registry \\ 
   & & Environment:  & Detector provider familiar with Docker, gRPC and detector proto file\\ 
   & & Response:  & New detector is added\\ 
   & & Response Measure:  & No more than five person-days of detector provider team effort is required for the implementation (legal and financial agreements are not included).\\ \hline 
 2 & Modifiability & Source:  & Camera Docker Image Provider\\
   & & Stimulus:  & Add a new camera to the vision system\\ 
   & & Artifact:  &  Camera \\ 
   & & Environment:  & Camera provider familiar with Docker, gRPC and camera proto file\\ 
   & & Response:  & New camera is added\\ 
   & & Response Measure:  & No more than four person-days of camera provider team effort is required for the implementation (legal and financial agreements are not included).\\ \hline
 3 & Modifiability & Source:  & RG Client\\
   & & Stimulus:  & Add a new proto file version\\ 
   & & Artifact:  &  Camera or Detector \\ 
   & & Environment:  & A new version of a proto file is necessary due to missing message types\\ 
   & & Response:  & New proto file is added\\ 
   & & Response Measure:  & No more than one person-day of effort is necessary. All existing camera and detector service remain functional.\\ \hline
 4 & Interoperability & Source  & Framework Architect\\
   & & Stimulus:  & Replace or scratch Docker\\ 
   & & Artifact:  & Detector and Camera Image Docker Containers\\ 
   & & Environment:  & Docker is not allowed or not possible to use\\ 
   & & Response:  & Docker is replaced or scratched\\ 
   & & Response Measure:  & No more than fifteen person-days of framework architect effort is required for the implementation\\ \hline
 5 & Performance & Source  & Detector Docker Image Provider\\
   & & Stimulus:  & Send a file to detector\\ 
   & & Artifact:  & Detector Docker Container\\ 
   & & Environment:  & CAD file input not larger than 10\,MB sent to detector\\ 
   & & Response: & File to detector sent\\ 
   & & Response Measure:  & The trip time for a request from detector client in the local network to detector during normal operation is less than 2 seconds.\\ \hline
 6 & Reliability & Source  & Framework Architect\\
   & & Stimulus  & Service failure\\ 
   & & Artifact  & All components\\ 
   & & Environment  & A sudden failure occurs in the runtime environment of a service.  \\ 
   & & Response  & After recovery, all transactions are completed or rolled back as appropriate, so the system maintains uncorrupted, persistent data.\\ 
   & & Response Measure:  & Manual test if data is corrupt or inconsistent.\\ \hline
 7 & Security & Source  & A possible attacker\\
   & & Stimulus  & A malware is included in the system\\ 
   & & Artifact  & Camera or Detector Docker Container\\ 
   & & Environment  & We have to face a denial-of-service attack.  \\ 
   & & Response  & The malware is detected quickly, best before entering the framework. Necessary information to possibly identify the attacker is logged.\\ 
   & & Response Measure:  & Our system is not operable for not more than an hour.\\ \hline
\end{longtable}

\subsection{Architectural Analysis of Scenarios}

\begin{longtable}{| P{0.2\textwidth} | P{0.74\textwidth} |}
\caption{Architectural Analysis of Scenarios}\label{tab:scenan}\\
\hline
\endfirsthead
\multicolumn{2}{c}%
{\tablename\ \thetable\ -- \textit{Continued}} \\
\hline
\endhead
\hline \multicolumn{2}{r}{\textit{Continued on next page}} \\
\endfoot
\hline
\endlastfoot
\rowcolor{Gray}
\multicolumn{2}{ |l| }{\textbf{Analysis for Scenario 1}} \\ \hline
Scenario Summary & New Detector added to Docker Registry in no more than five person-days\\ \hline
Business Goal(s) & Extending the available object detection possibilities\\ \hline
Quality Attribute & Modifiability\\ \hline
Architectural Approaches and Reasoning & Detector communication depends on gRPC which covers many programming languages and a REST gateway. \newline Docker registry can be made easily accessible for the provider and is a well-known technology.\\ \hline
Risks &  Possible too much of a lock-in effect of both technologies\\ \hline
Tradeoffs &  The static Train/Test interface might not (yet) be suitable for all detection purposes, although once implemented it is very easy to use and might become more popular as a versioned standard.\\ \hline
\rowcolor{Gray}
\multicolumn{2}{ |l| }{\textbf{Analysis for Scenario 2}} \\ \hline
Scenario Summary & New camera added to vision system in no more than four person-days\\ \hline
Business Goal(s) & Possibility of heterogeneous camera ecosystem\\ \hline
Quality Attribute & Modifiability\\ \hline
Architectural Approaches and Reasoning & Docker container including or alternatively just gRPC server running on camera which allows for many programming languages \\ \hline
Risks &  The camera firmware can be unmodifiable, thus a middleware is likely to be needed. \\ \hline
Tradeoffs & In theory, the same underlying technology (Docker + gRPC) can be conveniently used for communication between recipe and camera, but in practice, at least for the camera, this might end up cumbersome without a fitting middleware. \\ \hline
\rowcolor{Gray}
\multicolumn{2}{ |l| }{\textbf{Analysis for Scenario 3}} \\ \hline
Scenario Summary & A new proto file version is added in no more than on person-day.\\ \hline
Business Goal(s) & Staying up to date with current ODMs\\ \hline
Quality Attribute & Modifiability\\ \hline
Architectural Approaches and Reasoning & gRPC offers a convenient way of expanding proto files. As long as no messages are deleted or overwritten, the system remains downward compatible.\\ \hline
Risks &  None\\ \hline
Tradeoffs &  The more messages are created, the more overhead will be sent over the channels.\\ \hline
\rowcolor{Gray}
\multicolumn{2}{ |l| }{\textbf{Analysis for Scenario 4}} \\ \hline
Scenario Summary & Replace or scratch Docker in no more than 15 person-days.\\ \hline
Business Goal(s) & Permit easy integration with new business partners. \\ \hline
Quality Attribute & Interoperability\\ \hline
Architectural Approaches and Reasoning & Detector and Camera gRPC servers/clients do not need Docker for successful communication. It offers a huge convenience for sharing them, starting them, handling dependencies, etc. A switch to e.g. Heroku would require a partial rewrite of the framework.\\ \hline
Risks &  Too much effort included rewriting the framework. Instead, an alternative framework might be used.\\ \hline
Tradeoffs & If Docker is available, the ease of service administration is very high. If not, the flexibility is higher but requires more effort.\\ \hline
\rowcolor{Gray}
\multicolumn{2}{ |l| }{\textbf{Analysis for Scenario 5}} \\ \hline
Scenario Summary & A 10\,MB file is sent to detector in less than 2 seconds. \\ \hline
Business Goal(s) & Compatible message transmission speed.\\ \hline
Quality Attribute & Performance\\ \hline
Architectural Approaches and Reasoning &  gRPC sends its data via protobufs which are a binary format specialized in high-performance serialization and deserialization. If an image has to be transferred between OPC UA Server and Client, then the emerging Time Sensitive Networking will help facing real-time transmission challenges.\\ \hline
Risks & If too slow, the process might the bottleneck in production.\\ \hline
Tradeoffs &  None\\ \hline
\rowcolor{Gray}
\multicolumn{2}{ |l| }{\textbf{Analysis for Scenario 6}} \\ \hline
Scenario Summary & Uncorrupted, persistent data remains on the system after a service failure.\\ \hline
Business Goal(s) & Industrial-grade reliability \\ \hline
Quality Attribute & Reliability\\ \hline
Architectural Approaches and Reasoning & The framework relies on service choreography and client/server communication. Thus, every component has to ensure uncorrupted and persistent data. In the future, a messaging broker could be introduced for better reliability.\\ \hline
Risks & Currently, no reliability is granted, so a service failure will result in a failure of most components of the system \\ \hline
Tradeoffs & Adding a message broker adds reliability but also complexity. \\ \hline
\rowcolor{Gray}
\multicolumn{2}{ |l| }{\textbf{Analysis for Scenario 7}} \\ \hline
Scenario Summary & A denial-of-service attack causes no more than an hour of downtime of the framework.\\ \hline
Business Goal(s) & Offering a secure framework with little downtime. \\ \hline
Quality Attribute & Security\\ \hline
Architectural Approaches and Reasoning & The Docker Hub is intended to be accessible by many vendors to ensure a variety of detectors and cameras. Every new vendor has to ensure its integrity. New detectors and cameras have to be tested thoroughly for a possible malware function. Through the standardized interface, a malfunctioning service can best be replaced quickly by an equivalent one offered by another vendor.\\ \hline
Risks &  A service that is relied on in production reveals itself as malware can be hard to replace in the given time of one hour. Redundant services by multiple vendors can help out in this case.\\ \hline
Tradeoffs & Due to possibly high numbers of services and creative attackers testing every new vendor and service can be a tedious task.\\ \hline
\end{longtable}
}