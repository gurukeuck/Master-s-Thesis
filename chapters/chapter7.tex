\chapter{Conclusion and Future Research Directions\label{cha:chapter7}}
\section{Conclusion}
The aim of this thesis was to automatically generate ODSs with varying detection methods and interfaces. Hence, this results in three challenges:
\begin{itemize}
    \item Generating ODS automatically
    \item Handling ODS with varying detection methods
    \item Handling ODS with varying interfaces
\end{itemize}
In response, a SOA concept which tackles these challenges was created. Generating ODS (semi-) automatically has been undergird with a supporting framework. It helps to combine the necessary components of camera image acquisition, ODMs and returning the pose in a standardized way. These services must adhere to a versioned semantic "Train" and "Test" definition which is generic enough to comprehend a broad set of ODMs. Expanding the semantic to a new version is of little effort.

The service interfaces are currently limited to gRPC or REST. Albeit this covers two established interfaces, they lack certain functionality such as message queuing and real-time functionalities. A more holistic implementation is desirable for the future, e.g. with OPC UA over TSN. This would allow for even greater interoperability and reliability.

\section{Future Research Directions}
Further research directions include fulfilling all OPC UA Vision specification aspects once they are fully released. Especially the drill down to component level and a specification of how to interface cameras will offer the possibility for high-speed implementation of new components.

Also, the framework could be upgraded with smart aspects such as choosing the right camera for a detection task or automated training when a new product type is produced. With the help of these features, recipes could be generated automatically.

A very user-friendly way of extending the framework would be utilizing a catalogue of ODMs and possible components. Examples of these catalogues can be found on Eclipse Vorto~\cite{Eclipse2019Vorto2019} or the "yellow pages" for services~\cite{Kretschmer2014TeilnehmerverwaltungSteuerungsplattform}.