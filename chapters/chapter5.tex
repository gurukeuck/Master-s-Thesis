\chapter{Ergebnisse\label{cha:chapter5}}
Im folgendem werden die Ergebnisse des Projekts anhand der gestellten Anforderungen aus Abb. \ref{table:anforderungen} präsentiert. Die Betrachtung erfolgt Abschnittsweise mit den Meilensteinen als Gliederungspunkten.


\section{Objekterkennung}
Von den Objekterkennungsdiensten haben wir uns mehrere in Betracht gezogen, insbesondere Google Cloud Vision, Microsoft Azure Cognitive Services und Amazon Rekognition. Erfolgreich implementiert wurden die Templates für Google Cloud Vision und Amazon Rekognition. Nachdem die Authentifications-Files in der Datenbank hinterlegt wurden ist es möglich direkt oder aus der Datenbank Bilder an den oder die Services zu schicken und man erhält die Rückgabe als JSON. Damit sind die von uns geforderten Anforderungen für diesen Bereich alle erfüllt.
Die beiden Objekterkennungsdienste liefern nur Labels der erkannten Objekte. Für die Positionen der Objekte kann als Ausblick beispielsweise das von uns gebaute Tensorflowmodell als Webservice eingerichtet werden.


\section{Kamera}
Bei der Orbbec Persee handelt es sich um eine Kamera mit ARM-Prozessor. Es traten dadurch Probleme mit der Installation von OpenCV auf, da OpenCV standardmäßig nicht für ARM-Prozessoren ausgelegt ist . Für die zukünftige Verwendung muss OpenCV installiert werden. Der Quellcode für das Bildaufnehmen befindet sich schon auskommentiert im Programm. \newline
Aufgrund des wenig vorhandenen Speicherplatzes ist es nicht möglich das Tensorflow-Modell lokal zu betreiben. Tensorflow muss als Webservice implementiert werden und über das Netzwerk angesprochen werden können.

\section{Konfigurationswerkzeug für die Kamera}
Als Framework für die Web-Anwendung wird Django mit Python 3 verwendet. Als Alternative wurde Flask betrachtet, allerdings haben wir uns aufgrund von ausgezeichneter Dokumentation, großer Community und dem ebenfalls beliebten Django REST Framework für Django entschieden.
Die Funktionen für das Senden der Bilder und Empfangen der Antwort funktioniert mit und ohne Hinterlegung der Daten in einer Datenbank. Bilder hinzufügen und auch das Aufnehmen der Bilder über die Kamera ist für den Nutzer über die Oberfläche möglich und leicht verständlich. Sämtliche Funktionen funktionieren über die grafische Oberfläche, lassen sich aber auch per REST-Befehl nutzen. Services können hinzugefügt werden, dafür existieren zwei Templates für Google und für Amazon. Weiterhin besteht die Möglichkeit, andere oder eigene Services durch eine URL, einem Payload mit zusätzlichen Parametern und einem Authentifications-File zu ergänzen.
Die Antworten werden getrennt betrachtet und nicht ausgewertet.

\section{Tests}
Sämtliche integrierte Funktionen wurden manuell überprüft und auf Fehler getestet. Ein automatisiertes Testen mithilfe von Django unittests kam nicht zustande. Durch die integrierte Validierung werden großteilig fehlerhafte und ungewollte Eingaben durch den Nutzer verhindert und er wird auf den Fehler hingewiesen.