\thispagestyle{empty}
\vspace*{1.0cm}

\begin{center}
    \textbf{Abstract}
\end{center}

\vspace*{0.5cm}

\noindent
Contemporary manufacturing relies on monolithic object detection (OD) software architecture to achieve high-performance automation. As production machines typically have a life span of over 20 years, object detection aligns to this rigid pattern. Typically, necessary OD hardware is delivered along with the software. In the last decade, OD research leaped forward while product life cycles shortened concurrently. Thus, it is desirable to get more frequent updates. Subsequently, the goals of this thesis are eliminating hardware and saving maintenance costs while simultaneously keeping up to date with state of the art OD methods. This thesis introduces \textit{Recipe Generator}, a framework based on service-oriented architectures. It trains OD services stored in a Docker hub using a computer-aided design file. Then, it combines pairs of trained OD services and camera image acquisition services to recipes. Recipes are transferred, prepared and executed via an open platform communication unified architecture (OPC UA) vision server for easy manufacturing integration. The services are dockerized and communicate via Google remote procedure calls or representational state transfer. The main advantages of this approach are the flexible reuse of existing OD methods and the ease to add new ones in a preferred language and platform. There are two main challenges. The first is to increase the performance, reliability and availability of the framework by, e.g., time-sensitive networking. The second is to enhance the framework with a catalogue for "plug-and-play" services.