\thispagestyle{empty}
\vspace*{1.0cm}

\begin{center}
    \textbf{Abstract}
\end{center}

\vspace*{0.5cm}

\noindent
Contemporary manufacturing relies on monolithic object detection (OD) software architecture to achieve high-performance automation. As production machines typically have a life span of over 20 years, object detection aligns to this rigid pattern. Typically, necessary OD hardware is delivered along with the software. In the last decade, OD research leaped forward while product life cycles shortened concurrently; thus it is desirable to get more frequent updates. Subsequently, the goals of this thesis are eliminating hardware and saving maintenance costs while simultaneously keeping up to date with state of the art OD methods (ODM). I introduce \textit{Recipe Generator}, a framework based on service-oriented architectures (SOA). It trains  OD services (ODS) stored in a Docker Hub with a Computer Aided Design (CAD) file. Then, it combines pairs of trained ODS and camera image acquisition services to recipes. Recipes are transferred, prepared and executed via an Open Platform Communication Unified Architecture (OPC UA) Vision Server for easy manufacturing integration. The services are dockerized and communicate via Google Remote Procedure Calls (gRPC) or Representational State Transfer (REST). The main advantages of this approach are the flexible reuse of existing ODMs and ease to add new ones in a preferred language and platform. The main challenges are to increase the reliability and availability of the framework by, e.g., an Enterprise Service Bus (ESB).