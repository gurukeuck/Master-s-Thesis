\appendix
\chapter{Roles}
\label{sec:involvedparties}
\extrarowsep=0.6cm\begin{longtabu} to \linewidth{lX}
    Recipe Generator Client & The instance calling recipe generator methods Train, Test, Transfer, GenRecipe and GetConfig. It needs to have a certain intelligence to decide when to call which methods with the appropriate parameters. In the current state, this will most likely be a production engineer. It is assumed that adding or altering a recipe does not happen very frequently. In later use where more business intelligence resides inside recipe generator, the client could be a component, e.g. a PLC.\\ 
    Detector Image Provider & The instance developing new ODMs. This could be research teams, community projects like OpenCV or OD specialists of the company using recipe generator. For every new ODM, it is necessary to adhere to the recipe generator API or, for a new pattern of ODMs, update the recipe generator API.\\
    Camera Image Provider &  	The instance providing new camera hardware and capable software of interacting with recipe generator.\\ 
    Framework Architect & The author or contributor of recipe generator. He or she is responsible for the SOA design, continuous development and integration and updated proto files. Often the same person as the recipe genetor client.
\end{longtabu}


\chapter{Example of a recipe life cycle}
To illustrate recipe management, here is a possible life cycle of a recipe taken from the OPC UA Vision specification. \cite{VDMA2018OPCSpecification} Note that the method signatures are not necessarily exact here. 
\begin{enumerate}
    \item A recipe for ProductId-m is created externally (often centrally). 
    \item The recipe is pushed to the vision system with ExternalId-1, ProductId-m (using AddRecipe())
    \begin{itemize}
        \item It is stored there with ExternalId-1, InternalId-11.
        \item It is linked to ProductId-m on the vision system
    \end{itemize}
    \item There are further possible actions on the recipe without any particular order.
    \begin{itemize}
        \item The recipe may be edited locally later, keeping its ExternalId-1 and receiving InternalId-12.
        \item A (binary) different version of the recipe with the same ExternalId-1 may be pushed to the vision system later, receiving InternalId-13.
        \item The recipe may be linked later to ProductId-n on the vision system. Note that the external recipe 
        management does not concern itself with the InternalIds of the recipes on different vision systems. If 
        there are, due to one of these operations, several recipes on the vision system with identical 
        ExternalIds but different InternalIds, the vision system/server combination has no means of telling 
        which of these was targeted by the environment. It may choose to link all of them, or the newest one, 
        or the latest one pushed (ignoring internally edited ones). This is outside the scope of this specification.
    \end{itemize}
    \item The automation system is undergoing a change-over process to a specific product, namely ProductId-m. It 
    will re-tool the vision system
    \begin{itemize}
    \item by calling PrepareRecipe(ExternalId-1); the vision system then selects one of the existing recipes with 
    ExternalId-1 based on internal rules.
    \item by calling PrepareRecipe(ProductId-m); the vision system then selects one of the existing recipes 
    linked to ProductId-m based on internal rules. 
    \end{itemize}
    \item The vision system is commanded to process a specific product
    \begin{itemize}
        \item by calling StartJob(ExternalId-1); the vision system then starts processing with the recipe prepared for ExternalId-1. 
        \item by calling StartJob(ProductId-m); the vision system then starts processing with the recipe prepared for ProductId-m. 
        \item by calling StartJob(ExternalId-2); the vision system then throws an error because no such recipe has 
        been added or prepared.
        \item by calling StartJob(ProductId-p); the vision system then throws an error because no such recipe has 
        been added or prepared. 
    \end{itemize}
    \item If there is no error, the vision system carries out its task, going through the Executing state to return to state Ready waiting for further instructions. There are many other possibilities of errors, e.g. trying to prepare a recipe which is actually a sub-recipe, i.e., not capable of being processed by itself. 
\end{enumerate}

% \chapter{OPC UA Vision Method Signatures used in this Thesis}

\chapter{Proto Files}
\label{proto}
\section{Detector}
\lstinputlisting[language=protobuf3,style=protobuf]{img/detector.proto}

\section{Camera}
\lstinputlisting[language=protobuf3,style=protobuf]{img/camera.proto}

\endinput
