
\appendix
% \begin{appendices}

\chapter{Roles}
\label{sec:involvedparties}
\pagenumbering{roman}
\extrarowsep=0.6cm\begin{longtabu} to \linewidth{lX}
    Recipe Generator Client & The instance calling recipe generator methods Train, Test, Transfer, GenRecipe and GetConfig. It needs to have a certain intelligence to decide when to call which methods with the appropriate parameters. In the current state, this will most likely be a production engineer. It is assumed that adding or altering a recipe does not happen very frequently. In later use where more business intelligence resides inside recipe generator, the client could be a component, e.g. a PLC.\\ 
    Detector Image Provider & The instance developing new ODMs. This could be research teams, community projects like OpenCV or OD specialists of the company using recipe generator. For every new ODM, it is necessary to adhere to the recipe generator API or, for a new pattern of ODMs, update the recipe generator API.\\
    Camera Image Provider &  	The instance providing new camera hardware and capable software of interacting with recipe generator.\\ 
    Framework Architect & The author or contributor of recipe generator. He or she is responsible for the SOA design, continuous development and integration and updated proto files. Often the same person as the recipe genetor client.
\end{longtabu}


\chapter{Example of a Recipe Life Cycle}
\label{chap:recipelifecycle}
To illustrate recipe management, here is a possible life cycle of a recipe taken from the OPC UA Vision specification appendix~\cite{VDMA2018OPC40100-1:2018-11}. Note that the method signatures are not necessarily exact here. 
\begin{enumerate}
    \item A recipe for ProductId-m is created externally (often centrally). 
    \item The recipe is pushed to the vision system with ExternalId-1, ProductId-m (using AddRecipe())
    \begin{itemize}
        \item It is stored there with ExternalId-1, InternalId-11.
        \item It is linked to ProductId-m on the vision system
    \end{itemize}
    \item There are further possible actions on the recipe without any particular order.
    \begin{itemize}
        \item The recipe may be edited locally later, keeping its ExternalId-1 and receiving InternalId-12.
        \item A (binary) different version of the recipe with the same ExternalId-1 may be pushed to the vision system later, receiving InternalId-13.
        \item The recipe may be linked later to ProductId-n on the vision system. Note that the external recipe 
        management does not concern itself with the InternalIds of the recipes on different vision systems. If 
        there are, due to one of these operations, several recipes on the vision system with identical 
        ExternalIds but different InternalIds, the vision system/server combination has no means of telling 
        which of these was targeted by the environment. It may choose to link all of them, or the newest one, 
        or the latest one pushed (ignoring internally edited ones). This is outside the scope of this specification.
    \end{itemize}
    \item The automation system is undergoing a change-over process to a specific product, namely ProductId-m. It 
    will re-tool the vision system
    \begin{itemize}
    \item by calling PrepareRecipe(ExternalId-1); the vision system then selects one of the existing recipes with 
    ExternalId-1 based on internal rules.
    \item by calling PrepareRecipe(ProductId-m); the vision system then selects one of the existing recipes 
    linked to ProductId-m based on internal rules. 
    \end{itemize}
    \item The vision system is commanded to process a specific product
    \begin{itemize}
        \item by calling StartJob(ExternalId-1); the vision system then starts processing with the recipe prepared for ExternalId-1. 
        \item by calling StartJob(ProductId-m); the vision system then starts processing with the recipe prepared for ProductId-m. 
        \item by calling StartJob(ExternalId-2); the vision system then throws an error because no such recipe has 
        been added or prepared.
        \item by calling StartJob(ProductId-p); the vision system then throws an error because no such recipe has 
        been added or prepared. 
    \end{itemize}
    \item If there is no error, the vision system carries out its task, going through the Executing state to return to state Ready waiting for further instructions. There are many other possibilities of errors, e.g. trying to prepare a recipe which is actually a sub-recipe, i.e., not capable of being processed by itself. 
\end{enumerate}

% \chapter{OPC UA Vision Method Signatures used in this Thesis}
\chapter{ATAM Evaluation}
\section{Quality Attribute Scenarios}
For an explanaition of the sources (i.e., roles), see annex~\ref{sec:involvedparties}.
{\renewcommand{\arraystretch}{0.7} % vertical spacing between rows
\begin{longtable}{|P{0.08\linewidth}|P{0.2\linewidth}|P{0.16\linewidth} P{0.44\linewidth}|}
\caption{Quality Attribute Scenarios}\label{tab:scen}\\
\hline
\rowcolor{Gray}
\textbf{Number} & \textbf{Quality Attribute} & \multicolumn{2}{l|}{\textbf{Scenario}}\\
\hline
\endfirsthead
\multicolumn{4}{c}%
{\tablename\ \thetable\ -- \textit{Continued}} \\
\hline
\rowcolor{Gray}
\textbf{Number} & \textbf{Quality Attribute} & \multicolumn{2}{l|}{\textbf{Scenario}}\\
\hline
\endhead
\hline \multicolumn{4}{r}{\textit{Continued on next page}} \\
\endfoot
\hline
\endlastfoot
 1 & Modifiability & Source:  & Detector Docker Image Provider\\
   & & Stimulus:  & Add a new detector\\ 
   & & Artifact:  &  Docker Registry \\ 
   & & Environment:  & Detector provider familiar with Docker, gRPC and detector proto file\\ 
   & & Response:  & New detector is added\\ 
   & & Response Measure:  & No more than five person-days of detector provider team effort is required for the implementation (legal and financial agreements are not included).\\ \hline 
 2 & Modifiability & Source:  & Camera Docker Image Provider\\
   & & Stimulus:  & Add a new camera to the vision system\\ 
   & & Artifact:  &  Camera \\ 
   & & Environment:  & Camera provider familiar with Docker, gRPC and camera proto file\\ 
   & & Response:  & New camera is added\\ 
   & & Response Measure:  & No more than four person-days of camera provider team effort is required for the implementation (legal and financial agreements are not included).\\ \hline
 3 & Modifiability & Source:  & RG Client\\
   & & Stimulus:  & Add a new proto file version\\ 
   & & Artifact:  &  Camera or Detector \\ 
   & & Environment:  & A new version of a proto file is necessary due to missing message types\\ 
   & & Response:  & New proto file is added\\ 
   & & Response Measure:  & No more than one person-day of effort is necessary. All existing camera and detector service remain functional.\\ \hline
 4 & Interoperability & Source  & Framework Architect\\
   & & Stimulus:  & Replace or scratch Docker\\ 
   & & Artifact:  & Detector and Camera Image Docker Containers\\ 
   & & Environment:  & Docker is not allowed or not possible to use\\ 
   & & Response:  & Docker is replaced or scratched\\ 
   & & Response Measure:  & No more than fifteen person-days of framework architect effort is required for the implementation\\ \hline
 5 & Performance & Source  & Detector Docker Image Provider\\
   & & Stimulus:  & Send a file to detector\\ 
   & & Artifact:  & Detector Docker Container\\ 
   & & Environment:  & CAD file input not larger than 10\,MB sent to detector\\ 
   & & Response: & File to detector sent\\ 
   & & Response Measure:  & The trip time for a request from detector client in the local network to detector during normal operation is less than 2 seconds.\\ \hline
 6 & Reliability & Source  & Framework Architect\\
   & & Stimulus  & Service failure\\ 
   & & Artifact  & All components\\ 
   & & Environment  & A sudden failure occurs in the runtime environment of a service.  \\ 
   & & Response  & After recovery, all transactions are completed or rolled back as appropriate, so the system maintains uncorrupted, persistent data.\\ 
   & & Response Measure:  & Manual test if data is corrupt or inconsistent.\\ \hline
 7 & Security & Source  & A possible attacker\\
   & & Stimulus  & A malware is included in the system\\ 
   & & Artifact  & Camera or Detector Docker Container\\ 
   & & Environment  & We have to face a denial-of-service attack.  \\ 
   & & Response  & The malware is detected quickly, best before entering the framework. Necessary information to possibly identify the attacker is logged.\\ 
   & & Response Measure:  & Our system is not operable for not more than an hour.\\ \hline
\end{longtable}
\newpage

\section{Architectural Analysis of Scenarios}
\begin{longtable}{| P{0.2\textwidth} | P{0.74\textwidth} |}
\caption{Architectural Analysis of Scenarios}\label{tab:scenan}\\
\hline
\endfirsthead
\multicolumn{2}{c}%
{\tablename\ \thetable\ -- \textit{Continued}} \\
\hline
\endhead
\hline \multicolumn{2}{r}{\textit{Continued on next page}} \\
\endfoot
\hline
\endlastfoot
\rowcolor{Gray}
\multicolumn{2}{ |l| }{\textbf{Analysis for Scenario 1}} \\ \hline
Scenario Summary & New Detector added to Docker Registry in no more than five person-days\\ \hline
Business Goal(s) & Extending the available object detection possibilities\\ \hline
Quality Attribute & Modifiability\\ \hline
Architectural Approaches and Reasoning & Detector communication depends on gRPC which covers many programming languages and a REST gateway. \newline Docker registry can be made easily accessible for the provider and is a well-known technology.\\ \hline
Risks &  Possible too much of a lock-in effect of both technologies\\ \hline
Tradeoffs &  The static Train/Test interface might not (yet) be suitable for all detection purposes, although once implemented it is very easy to use and might become more popular as a versioned standard.\\ \hline
\rowcolor{Gray}
\multicolumn{2}{ |l| }{\textbf{Analysis for Scenario 2}} \\ \hline
Scenario Summary & New camera added to vision system in no more than four person-days\\ \hline
Business Goal(s) & Possibility of heterogeneous camera ecosystem\\ \hline
Quality Attribute & Modifiability\\ \hline
Architectural Approaches and Reasoning & Docker container including or alternatively just gRPC server running on camera which allows for many programming languages \\ \hline
Risks &  The camera firmware can be unmodifiable, thus a middleware is likely to be needed. \\ \hline
Tradeoffs & In theory, the same underlying technology (Docker + gRPC) can be conveniently used for communication between recipe and camera, but in practice, at least for the camera, this might end up cumbersome without a fitting middleware. \\ \hline
\rowcolor{Gray}
\multicolumn{2}{ |l| }{\textbf{Analysis for Scenario 3}} \\ \hline
Scenario Summary & A new proto file version is added in no more than on person-day.\\ \hline
Business Goal(s) & Staying up to date with current ODMs\\ \hline
Quality Attribute & Modifiability\\ \hline
Architectural Approaches and Reasoning & gRPC offers a convenient way of expanding proto files. As long as no messages are deleted or overwritten, the system remains downward compatible.\\ \hline
Risks &  None\\ \hline
Tradeoffs &  The more messages are created, the more overhead will be sent over the channels.\\ \hline
\rowcolor{Gray}
\multicolumn{2}{ |l| }{\textbf{Analysis for Scenario 4}} \\ \hline
Scenario Summary & Replace or scratch Docker in no more than 15 person-days.\\ \hline
Business Goal(s) & Permit easy integration with new business partners. \\ \hline
Quality Attribute & Interoperability\\ \hline
Architectural Approaches and Reasoning & Detector and Camera gRPC servers/clients do not need Docker for successful communication. It offers a huge convenience for sharing them, starting them, handling dependencies, etc. A switch to e.g. Heroku would require a partial rewrite of the framework.\\ \hline
Risks &  Too much effort included rewriting the framework. Instead, an alternative framework might be used.\\ \hline
Tradeoffs & If Docker is available, the ease of service administration is very high. If not, the flexibility is higher but requires more effort.\\ \hline
\rowcolor{Gray}
\multicolumn{2}{ |l| }{\textbf{Analysis for Scenario 5}} \\ \hline
Scenario Summary & A 10\,MB file is sent to detector in less than 2 seconds. \\ \hline
Business Goal(s) & Compatible message transmission speed.\\ \hline
Quality Attribute & Performance\\ \hline
Architectural Approaches and Reasoning &  gRPC sends its data via protobufs which are a binary format specialized in high-performance serialization and deserialization. If an image has to be transferred between OPC UA Server and Client, then the emerging Time Sensitive Networking will help facing real-time transmission challenges.\\ \hline
Risks & If too slow, the process might be the bottleneck in production.\\ \hline
Tradeoffs &  None\\ \hline
\rowcolor{Gray}
\multicolumn{2}{ |l| }{\textbf{Analysis for Scenario 6}} \\ \hline
Scenario Summary & Uncorrupted, persistent data remains on the system after a service failure.\\ \hline
Business Goal(s) & Industrial-grade reliability \\ \hline
Quality Attribute & Reliability\\ \hline
Architectural Approaches and Reasoning & The framework relies on service choreography and client/server communication. Thus, every component has to ensure uncorrupted and persistent data. In the future, a messaging broker could be introduced for better reliability.\\ \hline
Risks & Currently, no reliability is granted, so a service failure will result in a failure of most components of the system \\ \hline
Tradeoffs & Adding a message broker adds reliability but also complexity. \\ \hline
\rowcolor{Gray}
\multicolumn{2}{ |l| }{\textbf{Analysis for Scenario 7}} \\ \hline
Scenario Summary & A denial-of-service attack causes no more than an hour of downtime of the framework.\\ \hline
Business Goal(s) & Offering a secure framework with little downtime. \\ \hline
Quality Attribute & Security\\ \hline
Architectural Approaches and Reasoning & The Docker Hub is intended to be accessible by many vendors to ensure a variety of detectors and cameras. Every new vendor has to ensure its integrity. New detectors and cameras have to be tested thoroughly for a possible malware function. Through the standardized interface, a malfunctioning service can best be replaced quickly by an equivalent one offered by another vendor.\\ \hline
Risks &  A service that is relied on in production reveals itself as malware can be hard to replace in the given time of one hour. Redundant services by multiple vendors can help out in this case.\\ \hline
Tradeoffs & Due to possibly high numbers of services and creative attackers, testing every new vendor and service can be a tedious task.\\ \hline
\end{longtable}
}

\chapter{Proto Files}
\label{proto}
\section{Detector}
\lstinputlisting[language=protobuf3,style=protobuf]{img/detector.proto}

\section{Camera}
\lstinputlisting[language=protobuf3,style=protobuf]{img/camera.proto}

\endinput
% \end{appendices}
