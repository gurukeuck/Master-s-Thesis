\thispagestyle{empty}
\vspace*{0.2cm}

\begin{center}
    \textbf{Zusammenfassung}
\end{center}

\vspace*{0.2cm}

%\noindent 
Im Rahmen des automatisierungstechnischen Projekts wurde ein Konfigurationswerkzeug für die Orbbec Persee 3D-Kamera entwickelt. Die mit dem Python Web-Framework Django entwickelte Schnittstelle des Werkzeugs ist eine REST Schnittstelle mit optionalem graphischen Bootstrap-Frontend. Der Nutzer erhält die Möglichkeit, über HTTP-Aufrufe oder über das graphische Frontend die Kamera anzusprechen, dort die Schnittstellen und Authentifizierungsdaten verschiedener Cloud Services zu hinterlegen, anschließend Bilder mit der Kamera aufzunehmen und diese an die konfigurierten Cloud Services zu schicken. Die Objekterkennung wird wegen hohen Rechenaufwands an diese ausgelagert. Die Antworten der verschiedenen Dienste können je nach Bedarf auf dem Speicher der Kamera hinterlegt oder lediglich einmalig angezeigt werden. Zur Verwaltung der Daten wird eine SQLite Datenbank genutzt. 

Zusätzlich wurde ein Tensorflow-Modell erstellt, da große Cloud Anbieter wie \emph{Google Cloud Vision} oder \emph{Amazon Rekognition} Objekte in Bildern zwar erkennen aber nicht lokalisieren können. Das Modell sollte auch als Cloud Service implementiert werden, wurde aufgrund geringer Priorisierung zugunsten anderer Arbeitspakete aber fallen lassen.

Das Konfigurationswerkzeug bietet hohes Automatisierungspotential und eine nutzerfreundliche grafische Oberfläche, was dem Standard aktueller Web Technologien entspricht und den Grundstein für die intelligente Kamera der Zukunft legt.