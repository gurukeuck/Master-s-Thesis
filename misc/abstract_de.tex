\thispagestyle{empty}
\vspace*{1.0cm}

\begin{center}
    \textbf{Zusammenfassung}
\end{center}

\vspace*{0.5cm}

\noindent

Die moderne Fertigung setzt auf monolithische Softwarearchitekturen in der  Objekterkennung (OE), um eine leistungsstarke Automatisierung zu erreichen. Da Produktionsmaschinen in der Regel eine Lebensdauer von über 20 Jahren haben, richtet sich die Objekterkennung nach diesem starren Muster. In der Regel wird die erforderliche OE-Hardware zusammen mit der Software geliefert. In den letzten zehn Jahren hat die OE-Forschung einen Sprung nach vorne gemacht, während sich gleichzeitig die Produktlebenszyklen verkürzt haben. Daher ist es wünschenswert, häufigere Aktualisierungen zu erhalten. Ziel dieser Arbeit ist es, Hardware zu eliminieren, Wartungskosten zu senken und gleichzeitig auf dem neuesten Stand der OE-Methoden zu bleiben. Diese Arbeit stellt \textit{Recepy Generator} vor, eine Rahmenstruktur (Framework), das auf Dienste-orientierten Architekturen basiert. Es trainiert OE-Services, die in einem Docker-Hub gespeichert sind, mit einer computergestützten Designdatei (CAD). Anschließend werden Paare von trainierten OE-Diensten und Bilderfassungsdiensten zu Rezepten kombiniert. Die Rezepte werden über einen OPC UA Vision-Server in die Fertigung übertragen und dort vorbereitet und ausgeführt. Die Dienste sind dockerisiert und kommunizieren über Googles Fernprozeduraufrufe (gRPC) oder REST. Die Hauptvorteile dieses Ansatzes sind die flexible Wiederverwendung vorhandener OE-Methoden und das einfache Hinzufügen neuer Methoden in einer bevorzugten Sprache und Plattform. Es gibt zwei Hauptherausforderungen. Die erste besteht darin, die Leistung, Zuverlässigkeit und Verfügbarkeit des Frameworks durch z.B. zeitsensitive Vernetzung (TSN) zu erhöhen. Die zweite besteht darin, das Framework mit einem Katalog für sofort betriebsbereite Dienste zu erweitern.